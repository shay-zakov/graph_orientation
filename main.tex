\documentclass[a4paper,12pt]{article}
%\usepackage[utf8]{inputenc}
\usepackage{multirow}
\usepackage{amsmath}
\usepackage{amsthm}
\usepackage{graphicx}


% Definintg \devc for double-arrow vector notation:
\newcommand\shrinkage{2.1mu}
\newcommand\vecsign{\mathchar"017E}
\newcommand\dvecsign{\smash{\stackon[-1.95pt]{\mkern-\shrinkage\vecsign}{\rotatebox{180}{$\mkern-\shrinkage\vecsign$}}}}
\newcommand\dvec[1]{\def\useanchorwidth{T}\stackon[-4.2pt]{#1}{\,\dvecsign}}
\usepackage{stackengine}
\stackMath
%\usepackage{graphicx}

% Commands for annotating path directions - either entire path or just last edge
% TODO: workout the appropriate arrow styls

\newcommand{\vecP}{\vec{P}}
\newcommand{\vecO}{\vec{O}}
\newcommand{\vecG}{\vec{G}}
\newcommand{\dir}[1]{\stackrel{\rightarrow}{#1}}
\newcommand{\dirO}[1][O]{\stackrel{\rightarrow}{#1}}
%\newcommand{\rdir}[1]{\rho ({#1})}
%\newcommand{\ldir}[1]{\lambda ({#1})}
%\newcommand{\rdir}[1]{\vec{#1}}
%\newcommand{\ldir}[1]{\cev{#1}}

\newcommand{\rdir}[1]{\ensuremath{\vec{#1}}} % All edges pointing right
\newcommand{\ldir}[1]{\ensuremath{\cev{#1}}} % All edges pointing left
\newcommand{\rlast}[1]{\ensuremath{#1^\succ}} % Last edge pointing right
\newcommand{\llast}[1]{\ensuremath{#1^\prec}} % Last edge pointing left
%\newcommand{\elr}[1]{\ensuremath{#1^\leftrightarrow}} % Last edge pointing either right or left
%\newcommand{\alr}[1]{\ensuremath{\dvec{#1}}} % All edges pointing either right or left
\newcommand{\optIx}[1][j]{\ensuremath{i^\star_{#1}}} % Last edge pointing left
\newcommand{\tildeb}[1]{\stackrel{\sim}{\smash{#1}\rule{0pt}{1.1ex}}}



\usepackage[linesnumbered,ruled,vlined,scleft]{algorithm2e}
%\DontPrintSumicolon % Some LaTeX compilers require you to use \dontprintsemicolon instead
\SetKw{Break}{break}
\SetKw{And}{and}
\SetKwComment{Precondition}{\textbf{precondition:~}}{}
\SetKwComment{Postcondition}{\textbf{postcondition:~}}{}
\SetKwComment{LoopInv}{\textbf{loop invariant:~}}{}
\SetKwComment{Condition}{\textbf{condition:~}}{}


\newtheorem{theorem}{Theorem}[section]
\newtheorem{proposition}{Proposition}[section]
\newtheorem{corollary}{Corollary}[section]
\newtheorem{lemma}{Lemma}[section]
\newtheorem{observation}{Observation}[section]
\newtheorem{claim}{Claim}[section]
\newtheorem{property}{Property}[section]
\newtheorem{problem}{Problem}[section]
\newtheorem{definition}{Definition}[section]
\newtheorem{remark}{Remark}[section]


%\makeatletter
%\g@addto@macro\algorithm{\setcounter{cond}{0}}
%\makeatother   

\makeatletter
\DeclareRobustCommand{\cev}[1]{%
	{\mathpalette\do@cev{#1}}%
}
\newcommand{\do@cev}[2]{%
	\vbox{\offinterlineskip
		\sbox\z@{$\m@th#1 x$}%
		\ialign{##\cr
			\hidewidth\reflectbox{$\m@th#1\vec{}\mkern4mu$}\hidewidth\cr
			\noalign{\kern-\ht\z@}
			$\m@th#1#2$\cr
		}%
	}%
}
\DeclareMathOperator*{\argmin}{arg\,min}
\makeatother

\begin{document}
\sloppy
	

\author{}
\title{Orienting an Edge-Bi-Weighted Graph to Minimize the Heaviest Path}
\date{}



\maketitle
%\tableofcontents

\section{Introduction}

A \emph{edge bi-weighted graph} is an undirected graph $G = (V, E)$, such that each edge $\{u, v\} \in E$ has a pair of (possibly different) weights $w(u, v)$ and $w(v, u)$ associated with its two possible orientations $(u, v)$ and $(v, u)$, respectively. An \emph{orientation} 
%$ \vec{O}$ 
of $G$ creates a directed graph $\vec{G}
%=\vec{O} (G)
$ 
%from $G$ 
by selecting for each undirected edge $\{u, v\} \in E$ exactly one of its two possible orientations. Denote by $\mathcal{O}(G)$ the set of all orientations of $G$. 

The problem we address is the following.

\noindent {\bf Input}: An edge bi-weighted graph $G$.

\noindent {\bf Output}: An orientation  of $G$ that minimizes the weight of 
the heaviest of the resulting simple directed paths.

Actually, there remains an ambiguity in the specification of the cost function because it 
fails to specify whether the minimum is taken only over maximal 
simple paths or over all simple subpaths. To distinguish between these two possibilities we define two measures.

\begin{definition}
	Given a directed path $\vec{P}=<v_0,\ldots,v_n>$ let 
	\begin{equation}\label{e.h}
		W(\vec{P})=\sum_{k=0}^{n-1} w(v_k,v_{k+1}).
	\end{equation}
	We define two cost measures for an oriented graph $\vec{G}$:
	\begin{align} 
		W_s(\vecG) &= \max\{W(\vec{P})\mid \vecP \mbox{ is a simple path in }\vecG \}, \\
		W_m(\vecG) &= \max\{W(\vec{P})\mid \vecP \mbox{ is a maximal simple path in }\vecG \}.
	\end{align}
	The corresponding cost functions for orienting an undirected graph are 
	\begin{align}
		W_s(G) &= \min\{W_s(\vecG)\mid \vecG \in \mathcal{O}(G)\}, \\
		W_m(G) &= \min\{W_m(\vecG)\mid \vecG \in \mathcal{O}(G)\}.
	\end{align}
\end{definition}

 To illustrate the differences between the two cost functions consider\\
 $\vec{P}=<v_0,v_1,v_2,v_3>$, with $w(v_0,v_1)=2, w(v_1,v_2)=-3,w(v_2,v_3)=6$.
 Then $W_m(\vec{P})=5, W_s(\vec{P})=6$.

A very useful property of $W_s$ is that it is monotone: the cost of a graph is never less than the cost of a subgraph.
\begin{lemma}\label{lem:sprop}
	Given any edge bi-weighted graph G,
	$W_{s}(G)\geq  W_{s}(G')$ for any subgraph $G'$ of $G$. 
\end{lemma}

\noindent {\bf Proof:}
It is clear from the definition (\ref{e.h}) of $W_s$ that $W_s(\vecP)\geq W_s(\vecP ')$for any subpath 
$\vecP '$ of $\vecP$.  
\qed

This property sets $W_m$ apart from $W_s$,  as the above example shows:
$W_{m}(\vecP)=5 <  W_{m}(\vecP')=6$ for the subgraph $\vecP '=<v_2,v_3>$ of $\vecP$.

However, if all weights are non-negative $W_m$, too, is monotonic. In fact, in that case the two definitions coincide.

\begin{lemma}
	If all of the weights an edge-bi-weighted graph $G$ are non-negative 
	then $W_{s}( \vecG )= W_{m}( \vecG )$ for any orientation $\vecG$.
	In particular,  $W_{s}(G)= W_{m}(G)$,  and 
	$\vecG$ is an optimal orientation of $G$ with respect to $W_{s}$
	if and only if it is optimal with respect to $ W_{m} $.
\end{lemma}
\noindent {\bf Proof:}
From the definitions of $W_m$ and $W_s$ it is clear that for any path graph P  $W_{m}(P)\leq  W_{s}(P)$, and that the inequality is in fact an equality if all the weights 
are non-negative. This implies $W_{s}(G)= W_{m}(G)$.

Clearly any orientation of $G$ that is optimal with respect to $W_m$ is also 
optimal with respect to $W_s$. 
\qed

\bigskip
{\bf Conjecture}:
\textit{Given an edge bi-weighted graph $G$, any orientation that is optimal with 
	respect to $W_{m}$ is also optimal with respect to $W_{s} $.}

\bigskip
In the coming sections we will consider the following two problems
for various classes of graphs.

\begin{problem}[
%	Minimizing the heaviest subpath - 
	HS]
	Given a class of edge bi-weighted graphs find an algorithm to compute
	$W_s(G) $
	for any $G$
	in the class.
\end{problem}

\begin{problem}[
%	Minimizing the heaviest maximal path - 
	HM]
	Given a class of edge bi-weighted graphs find an algorithm to compute $W_m(G) $
	for any $G$
	in the class.
\end{problem}

 \section{Algorithms for stars}\label{s.1}
 A $n$-star $S$ is a tree with one internal node $c$ and $n$ leaves.
 
% Say that u dominates v if both $w(c, u) \geq w(c, v)$ and $w(u, c)\geq w(v, c)$. When visualizing the leaves as a set of 2D points, the point corresponding to u dominates all points which are to the left and below it.
% 
% First, we can argue that we can exclude from consideration all vertices which are dominated by other vertices. For example, if v is dominated by u, then whatever the direction of {c, u} in the star orientation we chose the same direction for {c, v} and don't increase the overall score. This leaves us with a subset of dominating vertices (if two or more vertices have exactly the same in and out weights and are not dominated by any other vertex, we take one of them as a representative in the dominating set). When sorting this set according to non-decreasing out-weights, we get a sequence of strictly increasing out-weights and strictly decreasing in-weights (because no vertex in the set dominates another vertex in it). Let's denote this sequence of vertices by $u_1, u_2, ..., u_r$, and 
% add for convenience $u_0$ with $-\infty /+\infty$ out/in weights, and $u_{r+1}$ with 
% $\infty /-\infty$ out/in weights. 
% 
% Now, let's assume an optimal star orientation that includes $(c, u_j)$ for some $u_j$ in the dominating set. Clearly, we can orient all edges $\{c, u_i\}$ for $i < j$ as $(c, u_i)$ (since $w(c, u_i) < w(c, u_j)$). Similarly, if the orientation includes $(u_j, c)$ we can orient every $\{c, u_i\}$ with $i > j$ as $(u_i, c)$. Therefore, there exists some optimal orientation and a corresponding index $0 < j \leq r+1$ such for all $i < j$ we have the edges $(c, u_i)$, and for all $i \geq j$ we have $(u_i, c)$. The score of this orientation is $\max(w(c, u_{j-1}), w(u_j, c))$. So the algorithm can do a binary search over the sequence of dominating vertices and find an optimal star orientation.
% 
% For the first instance, finding and sorting the dominating set takes $O(n \log n)$ time, and then finding the optimal j takes an additional $O(\log n)$ time. When updating the instance by changing one edge weight, we just need to update the dominating set similarly as done for the linear graph $H_m$ algorithm in $O(log n)$ time, and again find the optimal $j$ in $O(\log n)$ time.

 \subsection{An algorithm for cost function $H_m$}
 \begin{algorithm}\label{a.starm}
 	\KwIn{a bi-weighted $n$-star $S$}
 	\KwOut{an optimal orientation of $S$ under $H_m$}
 	
Orient each edge $\{u,c\}$ inwards to \textit{c} if $w(u,c)<w(c,u)$,
and outward from $c$ otherwise\; \label{i0}
 	 
denote by $\tildeb{S}$ the resulting directed graph, by $BestCost$ its cost, 
by $E_{in}=\{(u_1,c),\ldots, (u_{\ell},c)\}$ the list of its inward edges, 
and by $E_{out}=\{(c,v_1),\ldots, (c,v_r)\}$ the list of its outward edges\;
\label{i00} reorder each of $E_{in}$ and $E_{out}$ so that
the weights of its edges are in non-increasing order\;

set $\tildeb{S'}$ to $\tildeb{S}$\;

\For {$k= 1$ to $\ell$}
  {update $\tildeb{S'}$ by flipping the direction of edge $(u_k,c)$, and update $BestCost$ if necessary;}
   \label{i1} 
set $\tildeb{S'}$ to $\tildeb{S}$\;
\For {$k= 1$ to $r$}
   {update $\tildeb{S'}$ by flipping the direction of edge $(c,v_k)$, and update $BestCost$ if necessary;}
  \label{i2}
 	\Return an orientation whose cost is $BestCost$\;
 	\caption{Algorithm BestOrientStar$_m (S)$}
 	\label{algo:os-s}
 \end{algorithm}

\bigskip

To establish the correctness of the algorithm we first prove its key observation:  
if initially  each edge is oriented
so that it points in its lighter direction then an optimal orientation can be
found by only flipping edges that were initially pointing inwards or only
flipping edges that were initially pointing outwards. 
\begin{lemma}\label{l.best}
	There is no optimal orientation in which both the largest inward edge weight is less than $w(u_1,c)$ 
	and the largest outward edge weight is less than $w(c,v_1)$.
\end{lemma}

\noindent \textbf{Proof:}
Note that the cost of the initial orientation is $h_m(\tildeb{S})=w(u_1,c)+w(c,v_1)$.
Suppose, by contradiction, that there is an optimal orientation $\tildeb{S}^*$ 
with largest inward weight $w(x,c)< w(u_1,c)$,
and largest outward weight $w(c,y)<w(c,v_1)$.
Then $h_m(\tildeb{S}^*)=w(x,c)+w(c,y)$.
The initial orientation implies that $w(c,u_1) > w(u_1,c)$, and $w(v_1,c) \geq w(c,v_1)$.
Since $\{u_1,c\}$ is an outward edge in $\tildeb{S^*}$ and $\{v_1,c\}$ an inward edge,
$h_m(\tildeb{S}^*)\geq  
w(v_1,c)+w(c,u_1) >  w(u_1,c)+w(c,v_1) =h_m(\tildeb{S})$, a contradiction.
\qed

\begin{theorem}\label{t.star}
%	[Correctness of the algorithm]
Algorithm \emph{BestOrientStar}$_m$ finds an optimal orientation in  $O(n \log n)$ steps.
\end{theorem}  

\noindent \textbf{Proof:}

 Let $\tildeb{S}^*$ be an optimal orientation of the input $n$-star $S$.
 
 If the edges in $\tildeb{S}^*$ are all oriented outwards, or all oriented 
 inwards, then the algorithm will find an optimal orientation
 at the completion of the loop of statement \ref{i1}, or
of statement \ref{i2}, respectively.
 
 Otherwise at least one of the edges $(u_1,c)$ and $(c,v_1)$ takes part 
 in $\tildeb{S}^*$, according to Lemma \ref{l.best}. If both participate then an optimal orientation is found in step \ref{i0}.
 If not, we consider and prove the case that $(c,v_1)$ participates. 
 
 Note first of all that if there is an inward edge in $\tildeb{S}^*$ that is oriented as
 an outward edge in $\tildeb{S}$ then flipping that edge cannot
 increase the cost of the orientation, because its outward weight does not exceed $w(c,v_1)$.
 We can assume, therefore, that all edges in $E_{out}$ participate in $\tildeb{S}^*$. 
 Let $(u_j,c)$ be the 
 edge from $E_{in}$ with least index that participates in $\tildeb{S}^*$, 
  $w(u_j,c)<w(u_1,c)$, i.e. all edges $\{u_i,c\}, 1\leq i<j$, are outward edges in $\tildeb{S}^*$. 
 The cost of $\tildeb{S}^*$ is therefore 
 $$h_m(\tildeb{S}^*)=w(u_j,c)+\max \{w(c,v_1),\max \{w(c,u_i):1\leq i<j)\}\}.$$
To complete the proof we observe that $h_m(\tildeb{S}^*)$ is precisely the value of the orientation
obtained by the algorithm after flipping the edge  $\{u_{j-1},c\}$
 in iteration $k=j-1$ of the loop of step \ref{i1}.

The running time  analysis is straightforward: the reordering of step \ref{i00} takes
$O(n \log n)$, and each update of $BestCost$ in loops \ref{i1} and \ref{i2} takes 
constant time if  auxiliary variables are maintained for the values of 
$\max \{w(c,u_i):1\leq i<j)\}$ and $\max \{w(v_i,c):1\leq i<j)\}$.
 \qed
 
  \subsection{An algorithm for cost function $H_s$}
  An algorithm for cost function $H_s$ is obtained by slightly modifying Algorithm BestOrientStar$_m$ using the following observation.
  
  \begin{lemma}
  	Only the weights of positive weight edges can contribute to the value of $h_s(\tildeb{S})$ for an oriented star $\tildeb{S}$.
  \end{lemma} 
\begin{proof}
	The length of a path in an oriented star is no more than 2. 
	An edge with non-positive weight is therefore either the first or 
	the last edge on any path, and does not contribute to its cost. 
\end{proof}

Here is the outline of the resulting algorithm for cost function $H_s$:
 \begin{algorithm}\label{a.stars}
	\KwIn{a bi-weighted $n$-star $S$}
	\KwOut{an optimal orientation of $S$ under $H_s$}	
	remove from $S$ every edge with a direction of non-positive weight, and denote the resulting star $S'$
	\label{s.rem} \;
	set $\tildeb{S'}$ to $\mbox{BestOrientStar}_m (S')$\;
	direct each edge removed in statement \ref{s.rem} in a direction of non-positive weight,
	and add it to $\tildeb{S'}$;
	denote the resulting directed star $\tildeb{S}$\;
	\Return $\tildeb{S}$\;
	\caption{Algorithm BestOrientStar$_s (S)$}
	\label{algo:oc-s}
\end{algorithm}

\section{Algorithms for linear graphs \label{s.2}}
We call a graph that has two vertices of degree 1 and all other vertices of degree 2  a \emph{ linear graph}, and 
reserve the term \emph{path} for a directed linear graph that has a single source (and a single sink).
Given a linear graph  $L$ on $n+1$ vertices number its vertices from \textit{0} to \textit{n}, and denote the weights of
edges $(i,i+1)$ and  $(i+1,i)$ by $w(i,i+1)$ and $w(i+1,1)$, respectively.

We describe next a generic algorithm for finding the cost of an optimal orientation 
of a bi-weighted linear graph. This high level version makes no use of the details
of the cost function, be it $H_m$ or $H_s$. Its description therefore uses the subscript $x$ with $x\in \{s,m\}$. However, the implementations under 
the two cost functions of this high level version, presented in the succeeding subsections, reveal surprising differences between their running times.

\subsection{A generic algorithm for linear graphs}

%\bigskip

{\bf Notation}:
\begin{itemize}
\item $L_{i, j}$ is the sub-graph of $L$ induced by the vertices $i,  \ldots, j$. 
\item $\vec{L}_{i, j}$ denotes the oriented version 
of $L_{i, j}$ in which all edges are directed to the right (i.e. of the form $(i, {i+1})$),
and $\cev{L}_{i, j}$ denotes the oriented version 
of $L_{i, j}$ in which all edges are directed to the left. 
\item $\rlast{H}_x(i)$ is the value of an optimal orientaton of $L_{i,n}$ under the constraint
that edge $\{i,i+1\}$ is directed towards $i+1$, and $H_x^{\prec}(i)$ is the value of an optimal orientaton of $L_{i,n}$ under the constraint
that edge $\{i,i+1\}$ is directed towards $i$.\\
In particular, the cost of an optimal orientation of $L$ under $H_x$ is 
$\min \{H_x^{\succ}(0),\ H_x^{\prec}(0)\}$.
\end{itemize}

The basic step of the recursive algorithm for finding an optimal orientation, or its cost, is to locate the last 
vertex of an optimal orientation of $L_{ j,n}$ at which there is a change in direction,
given that the very last edge has a specified direction. Keeping this in mind the proof 
of correctness of the algorithm is straightforward. 
For simplicity we outline the algorithm for finding the cost of an optimal orientation, BestCostLinear$_x(L)$.
That algorithm is easily converted into an algorithm for finding the orientation itself, BestOrientLinear$_x(L)$,
 by recording for each 
$H^{\succ}(j)$ and $H^{\prec}(j)$ which $i$ corresponds to the minimum in statements 
\ref{st.1} and \ref{st.2}.
%\bigskip

\begin{algorithm}
	\KwIn{a bi-weighted linear graph $L$ on $n+1$ vertices}
\KwOut{an optimal orientation of $L$ under $H_x$}
	$H^{\succ}(n)=H^{\prec}(n)=-\infty$, $n=length(L)+1$\;
	\For{$i=n-1$ to $0$}{
		set $H^{\succ}(i)$ to the minimum over $i< j \leq n$ of $\max \{h_x(\vec{L}_{i, j}), H^{\prec}(j)\}$\;
		\label{st.1}
		set $H^{\prec}(j)$ to the minimum over $ i <j \leq n$ of  $\max \{h_x(\cev{L}_{i, j}), H^{\succ}(i)\}$\;
			\label{st.2}
	}
	\Return{$\min \{H^{\succ}(0),\ H^{\prec}(0)\}$}\;
	\caption{BestCostLinear$_x(L)$}
	\label{algo:H}
\end{algorithm}
\begin{theorem}
Given input $L$, Algorithm \emph{BestCostLinear}$_x$ finds the cost of an optimal orientation of $L$.
\end{theorem}
A straightforward implementation of this algorithm runs in $O(n^2)$ time. 
The implementations described in the coming two subsections provide drastic improvements.

\subsection{Running time under cost function $H_s$}
In this subsection we show that 
Algorithm BestCostLinear$_s$ can be made to run in time $O(n)$ by carefully implementing
the computational operations of the algorithm.
Two issues have to be addressed.

	The first is the time needed to compute $h_s(\vec{L}_{i, j})$ 
	and $h_s(\cev{L}_{i, j})$ for given $0\leq i<j\leq n$.
	Focusing on $h_s(\vec{L}_{i, j})$, and combining equations (\ref{eq.W}) and (\ref{eq.hs}):
	\begin{equation}\label{eq.hsij}
	h_s(\vec{L}_{i, j})=\max \left\{ \sum_{t=i'}^{j'-1}w(t,t+1) \mid i\leq i' \leq j' \leq j\right\}.
	\end{equation}
	
	Computing $h_s(\vec{L}_{i, j})$ for a pair $(i,j)$ is therefore an instance
	of the Range Maximum-sum Segment On-line Query problem, 
	RMSOQ for short, as defined in \cite{chen2007range}:
	\begin{problem}[Range Maximum-sum Segment On-line Query problem]\ \\
		\noindent \emph{\bf Input to be preprocessed:}
		A nonempty sequence $a_1 ,\ldots a_n$ of real numbers.\\ 
		\noindent \emph{\bf Online query:} respond to a query of the form $RMSOQ( i, j)$ by returning a pair of indices $(i', j')$ 
		which maximize
		$\sum_{t=i'}^{j'}a_t$ over all $i\leq i' \leq j' \leq j$.	
	\end{problem}
	Chen and Chao \cite{chen2007range} presented a method for answering each such query 
	in constant time after  $A$ is preprocessed in $O(n)$ time. The following Lemma 
	summarizes the discussion and its relevance.
\begin{lemma}
	Suppose $w(i,i+1),0\leq i <n$ and $w(i+1,i),0\leq i <n$ have been preprocessed 
	in linear time for the RMSOQ problem.
	After  $H^{\prec}(i)$ and $H^{\succ}(i)$ have been computed for $0\leq i <j$ each 
	value of the form $\max \{ H^{\prec}(i), h_s(\vec{L}_{i, j})\}$ and $\max \{ H^{\succ}(i), h_s(\cev{L}_{i, j})\}$ appearing in statements \ref{st.1} and \ref{st.2} in iteration $j$
	can be evaluated in constant time.
\end{lemma}
A question remains: how many queries $RMSOQ( i, j)$ will the algorithm present?
The answer to this question will be found by addressing the second issue: what is the time needed to find all minimum values in statements \ref{st.1} and \ref{st.2}. Here the notion of a totally monotone matrix will be helpful. 
To emphasize that this monotinicity is rowwise we call it r-monitinicity.
\begin{definition}\label{d.tm}
	Given an $n \times n$ matrix $M$, denote
	by $C(i)$ the least column index at which the minimum in row $i$ of $M$ is achieved, 
	i.e.,
	$C(i) = \min \{k :M_{i,k} = min_{1\leq j \leq n} M_{i,j}\}$.
	The matrix $M$ is r-monotone if $C(1) \leq  C(2)\leq \cdots \leq C(n)$, and it is totally c-monotone 
	if all $2\times 2$
	submatrices of $M$ formed by choosing two rows and two columns are r-monotone.
\end{definition}
The following Lemma is easily verified.
\begin{lemma}\label{l.rmono}
	$M$ is totally r-monotone if and only if $M(i_1,j_1)\leq M(i_1,j_2)$
	whenever $M(i_2,j_1)\leq M(i_2,j_2)$,
for all $i_1<i_2$ and $j_1<j_2$.
\end{lemma}
Whereas for an arbitrary square matrix it takes quadratic time to 
compute all values $R(j)$, for a c-monotone matrix the SMAWK algorithm \cite{smawk1987} 
is able to do so, off-line, in linear time.
	
Define the matrices $M^{\succ}(i,j)$ and $M^{\prec}(i,j)$ as follows.
\begin{itemize}
	\item For $0\leq i<j \leq n$,
$$M^{\succ}(i,j)=\max \{h_s(\vec{L}_{i, j}), H^{\prec}(j)\},\ 
M^{\prec}(i,j)=\max \{h_s(\cev{L}_{i, j}), H^{\succ}(i)\}.$$	
	\item $M^{\succ}(i,j)=M^{\prec}(i,j)=\infty$  for $1\leq j \leq i \leq n $.
%	\item $M^{\succ}(0,j)=h_s(\vec{L}_{0, j})$, $M^{\prec}(0,j)=h_s(\cev{L}_{0, j})$, $1\leq j\leq n$.
\end{itemize}
In these terms the algorithm computes the minimum value in row $i$ of 
the matrices $M^{\succ}$
and $M^{\prec}$, for $0\leq i <n$.
Two features of this computation deserve particular attention.
\begin{enumerate}
	\item The computation of $M^{\succ}$
	and $M^{\prec}$ has an on-line flavor: before computing the minimum value in row $i$ of 
	the matrix $M^{\succ}$, or $M^{\prec}$, the minimum values of all rows $i<i'$ of 
	the matrix $M^{\prec}$, respectively $M^{\succ}$,  have to be available.
		\item \label{i.1} It follows from equation (\ref{eq.hsij}) that
	$h_s(\vec{L}_{i, j})$ and $h_s(\cev{L}_{i, j})$ are both non-increasing in $i$ 
	and non-decreasing in $j$. 
\end{enumerate}
According to the first feature, the problem of computing 
$H^{\succ}(j)$ and $H^{\succ}(j)$ can be dealt with by any method that solves the 
 following problem
\begin{problem}[ORM - Online Row Minima]
	For $1\leq i \leq n$ compute $H(i)=\min \{M(i,j) \mid 1\leq j \leq n\}$, where 
	the values of $H(i'),\ i< i'\leq n$ have to be computed before $M(i,j)$ can be evaluated.
\end{problem}
In our case, both $M^{\prec}$ and $M^{\succ}$ are of the form $M(i,j)=\max \{f(i,j),g(j)\}$,
where $f$ is non-increasing in $i$ and non-decreasing in $j$, according to the 
above second feature. This is the key to proving that $M^{\prec}$ and $M^{\succ}$ are totally r-monotone.
\begin{proposition}
	If $f$ is non-increasing in $i$ and non-decreasing in $j$ and $M(i,j)=\max \{f(i,j),g(j)\}$,
	then $M$ is totally r-monotone.
\end{proposition}
\begin{proof}
	According to Lemma \ref{l.rmono} we have to prove that for any $i_1<i_2$ and $j_1<j_2$,
	 $M(i_2,j_1)\leq M(i_2,j_1)$ implies $M(i_1,j_1)\leq M(i_1,j_2)$.

We claim that, for any $i$ and  $j_1<j_2$, 
\begin{equation}\label{e.iff}
M(i,j_1)\leq M(i,j_2) \mbox{ if and only if }  g(j_1)\leq \max \{ f(i, j_2),g(j_2)\}. 
\end{equation}
This follows from the definition of $M$, and the fact that 
 $f(i, j_1)\leq f(i, j_2)\leq \max \{ f(i, j_2),g(j_2)\} $ for $j_1 < j_2$.

Consequently,  if $M(i_2,j_1)\leq M(i_2,j_2)$ then
$$ g(j_1)\leq \max \{ f(i_2,j_2), g( j_2)\}\leq \max \{ f(i_1,j_2), g( j_2)\},$$
since $ f(i_2, j_2) \leq f(i_1, j_2)$.
Using equation (\ref{e.iff}) again, $M(i_1,j_1)\leq M(i_1,j_2)$.
\end{proof}
Summarizing the foregoing discussion, to find all minimum values in statements \ref{st.1} and \ref{st.2} of the algorithm we can employ the solution to the following problem.
\begin{problem}[ORMM - Online Row Minima of  a r-Monotone matrix]\label{p.ormm}
		For $1\leq j \leq n$ compute $H(j)=\min \{M(i,j) \mid 1\leq j \leq n\}$, where 
		$M$ is r-monotone and
	the values of $H(i'),\ i < i' \leq n$ have to be computed before $M(i,j)$ can be evaluated.
\end{problem}
Note that the SMAWK algorithm cannot
be used directly because it is off-line whereas our computation has to be online.
Fortunately, several linear-time online algorithms for solving Problem \ref{p.ormm} have been published, 
\cite{klawe89,larmore91,galil92,barnoy09}.

\begin{theorem}\label{t.linear-s}
	When the cost function used in Algorithm \emph{BestCostLinear}$_x$ is $H_s$ its running time is $O(n)$.
\end{theorem}

\

\subsection{Running time under cost function $H_m$}
We show here that 
Algorithm BestCostPath$_m$ can be made to run in time $O(n \log n)$.
As in the case of cost function $H_s$, the key is the analysis and careful implementation
of statements \ref{st.1} and \ref{st.2} of the algorithm.

As for computing $h_m(\vec{L}_{i, j})$, combining equations (\ref{eq.W}) and (\ref{eq.hm})
gives
\begin{equation}\label{eq.hmij}
h_m(\vec{L}_{i, j})=W_{j}-W_{i}= \sum_{t=i}^{j-1}w(t,t+1),
\end{equation}
where $W_{j}= \sum_{t=0}^{j-1}w(t,t+1)$, $W_0=0$.
A similar equality holds for $h_m(\cev{L}_{i, j})$.
Consequently, finding the minimum values in statements \ref{st.1} and \ref{st.2} takes on a form 
that is very different from the one obtained for $H_s$. Focusing on statement \ref{st.1}, 
equation (\ref{eq.hmij}) shows that for $1\leq j \leq n$

\begin{equation}
\begin{array}{ll@{}l}
H^{\succ}(j)
%&=\min_{0\leq i <j}\max &\{ H^{\prec}(i), h_m(\vec{L}_{i, j})\}\nonumber \\
&=\min_{0\leq i <j} \max &\{ H^{\prec}(i), W_{j}-W_{i} \}\nonumber.
\end{array}
\end{equation}
Let us split the points of the interval ${0\leq i <j}$ into those that satisfy 
$H^{\prec}(i)+W_{i}\geq W_{ j}$ and the remainder. Since $H^{\prec}(0)=-\infty$, 
the point $i=0$ belongs to the latter. In particular, $H^{\succ}(1)=W_1$, and for $j\geq 2$
$H^{\succ}(j)=\min \{M_1(j),W_{j}-M_2(j)\}$,  with 
\begin{align}
M_1(j)&=\min_{1\leq i <j } \{H^{\prec}(i) \mid 
       H^{\prec}(i)+W_{i}\geq W_{ j}
       \}, \label{e.min-m}\\
M_2(j)& =\min_{0\leq i <j} \{W_{ j}-W_{ i} \mid  
       H^{\prec}(i)+W_{i}<  W_{j}
       \}. \label{e.max-m}
\end{align}
We turn now to describing the data structures that enable the efficient computation of the minima in equation (\ref{e.min-m});  equation (\ref{e.max-m}) can be handled similarly. 
Rephrased slightly more abstractly we need to solve the following problem.

\begin{problem}[\emph{Minima of sequence prefixes under key-bounds}]\label{p.msp}\ 

\noindent \emph{\bf Given:} A sequence $KV$ of $n$ pairs of the form $(key,value)$,
	and a sequence of lower bounds $W_j,\  1\leq j \leq n$, 

\noindent \emph{\bf Compute:} 
\begin{equation}
min_j(W_j)=\min \{value \mid (key,value)\in Pre_j \mbox{ \emph{and} } key \geq W_j\},  \mbox{ for }1\leq j \leq n, \label{e.min-m-v}
\end{equation}
where $Pre_j$ is the prefix of length $j$ of $KV$.
\end{problem}

To solve this problem efficiently we will maintain a set, whose members are 
pairs  $s_i=(key_{i}, value_{i})$, sorted by $key_i$.  
Denote by $S_j$ the sorted set formed after $j$ pairs
from $KV$ have been processed.
For convenience we add a first pair $(key_{0},\mu_{0})=(-\infty,-\infty)$
and a last pair
$(key_{\ell_j+1},value_{\ell_j+1})=(\infty,\infty)$, with $\ell_j=|S_j|-2$.
All pairs in $S_j$, except for the first and last, are also present in $Pre_j$ but some of the 
latter's pairs may be absent from $S_j$. 

The operations on $S_j$ are:
\begin{itemize}
	\item  Initialization: Set $S_0$ to $<(-\infty,-\infty), (\infty,\infty)>$.
	\item  Computing $min_j(W_j)$:\\
	find the smallest $key_{k}$ in $S_{j}$ such that 
	$W_j\leq key_{k}$;\\  \textbf{return} $value_{k}$.\\
	\item Updating $S_{j-1}$ to $S_j$ using the $j$-th pair $s=(key, value)$ from $KV$:
	\begin{enumerate}
		\item let $key_{m}$ be the smallest key in $S_{j-1}$ such that 
		$key < key_{m}$;
		\item {\bf if} $value\geq value_{m}$ \textbf{then} $S_j $ is the same as $S_{j-1}$;\\
		\{\emph{comment:  s=$(key,value)$ } is discarded\} ; \label{i.discard}
		\item \textbf{else} \label{i.else}
		  \begin{enumerate}
		  	\item set $k$ to $m-1$;\\  \textbf{while} $value_k \geq value $ \textbf{do} \\
		  		delete $s_k$ from $S_{j-1}$; set $k$ to  $k-1$;\label{i.while}
		  	\item  create $S_j$ by inserting $ s$ into $S_{j-1}$ between $s_k$ and $s_m$;
		  	\label{i.insert}
		  \end{enumerate} 
	\end{enumerate}
\end{itemize}
\begin{figure}[h]
	\centering
	\begin{minipage}{0.45\textwidth}
		\centering  
		\includegraphics[width=0.9\textwidth]{KV_2.eps}
		\caption{Computing $min_j(W_j)$.}\label{fig:min_fig}
	\end{minipage}\hfill
	\begin{minipage}{0.45\textwidth}
		\centering 
		\includegraphics[width=0.9\textwidth]{KV_update_2.eps}
		\caption{Updating $S_j$}\label{fig:update_fig}
	\end{minipage}
\end{figure}

Figure  \ref{fig:min_fig} illustrates the computation of $min_j(W_j)$: the step function represents 
$S_j$, the scattered points are $(key, value)$ pairs that were discarded, and the points to the right
of the broken vertical line are the ones whose minimum value is to be found.

Figure \ref{fig:update_fig} illustrates the updating of $S_j$: the star point is a new pair that is added
to $S_j$, and $S_j$ is updated by discarding the points on the step function that are above the horizontal line to the left  of the star point.

\begin{theorem}\label{p.kmu} The updating procedure preserves the following invariants of $S_j$.
  \begin{enumerate}
  	\item If $0\leq i_1< i_2 \leq \ell_j+1$ then $key_{i_1}< key_{i_2}$ and 
  	$value_{i_1} < value_{i_2}$.
  	\item Suppose some $s'=(key',value') \in Pre_j$ does not appear in $S_j$, and
  	$key_{i-1} < key' \leq key_i$. Then $value'\geq value_i$.
   \end{enumerate}
   Taken together the two invariants imply the correctness of the $\min_j$ computation: if $key_{i-1}<W\leq key_{i}$ for some $1\leq i \leq \ell_j+1$. then 
  	$\min_j(W)=value_{i}$.
\end{theorem}
\begin{proof}
By induction on $j$. When $j=0$ both invariants are clearly true.

For general $j$, if $s=(key,value)$ was discarded in statement \ref{i.discard}, then the first invariant holds because it did for $S_{j-1}$, and the second invariant is correctly updated
by statement \ref{i.discard}. 

If $s$ is not discarded, $S_j$ is updated according to statement \ref{i.else}. In this case $S_{j-1}$ and $S_j$ differ only in that 
the pairs $s_{k+1},\ldots s_{m-1}$ were deleted from $S_{j-1}$ and $s=(key,value)$ was
inserted between $s_k$ and $s_m$, with $k$ and $m$ as in
update statement \ref{i.insert}. Since $key_k<key <key_m$ and 
$value_k < value <value_m$ the first invariant is maintained.
Moreover, all pairs $(key',value')$
deleted in statement \ref{i.while}
satisfy $key_k<key'\leq key$, $value'\geq value$, so that the second invariant is correctly
updated.
\end{proof}
\begin{theorem}\label{t.linear-m}
	When the cost function used in Algorithm \emph{BestCostPath}$_x$ is $H_m$ its running time is $O(n\log n)$.
\end{theorem}
\begin{proof}
One possible implementation of the sorted set data structure uses red-black trees, \cite{guibas}.
The operations on the data structure used in our solution of Problem \ref{p.msp} -  insert, delete, and search - each take $O(\log n)$ time, resulting 
in an overall running time of $O(n \log n)$.
\end{proof}



\section{An algorithm for cycles}\label{s.c}
Given a cycle graph $C_n$ on $n$ vertices, number its vertices from \textit{0} to \textit{n-1}, 
clockwise from an arbitrary node. 
Denote the weight of a directed
edge $(i,i+1)$ by $w(i,i+1)$. When a node $j$ with $j\geq n$ is referred to it 
should be understood as referring to node $j \mod n$. 
For example, $w(n-1,n)=w(n-1,0)$.

\begin{theorem}
	There is a linear time algorithm for finding an orientation for a cycle graph that is optimal	under the cost function $W_s$.
\end{theorem}
\noindent {\bf Proof}:

\qed
%The following algorithm uses the BestCostPath algorithm as a building block. 
%
%\noindent \textbf{Algorithm BestCostCycle ($C_n$)}:
%
%\begin{enumerate}
%	\item $BestCost \leftarrow \min\{\sum_{i=0}^{n-1}w(i,i+1), \sum_{i=0}^{n-1}w(i+1,i)\}$;
%	\item \label{ac.i1} for $i= 0$ to $n-1$ do:
%	\begin{enumerate}
%		\item break the cycle at vertex $i$ by making two copies of vertex 
%		$i$, $i'$ and $i''$, and setting $P$ as the path $i', i+1,\ldots ,0,\ldots i-1, i''$
%		with weights the same as in $C_n$ except for the edges involving $i', i''$;  
%		\item \label{i.c1}let $P^{out}$ be the path $P$
%		with $w(i',i+1)=w(i,i+1),\ w(i+1,i')=\infty$ and 
%		$w(i-1,i'')=\infty,\  w(i'',i-1)=w(i,i-1)$;
%		\item \label{i.c2}
%%		$Cost(P_1)\leftarrow BestCostPath(P_1)$; 
%		if $BestCostPath(P^{out})< BestCost$ then update $BestCost$;
%		\item \label{i.c3} let $P^{in}$ be the path $P$
%		with $w(i',i+1)=\infty,\ w(i+1,i')=w(i+1,i)$ and 
%		$w(i-1,i'')=w(i-1,i),\  w(i'',i-1)=\infty$;
%		\item \label{i.c4}
%%		$Cost(P_2)\leftarrow BestCostPath(P_2)$; 
%		if $BestCostPath(P^{in})< BestCost$ then update $BestCost$.
%	\end{enumerate}		
%	\item return $BestCost$; 
%\end{enumerate}	
%\noindent \textbf{End of Algorithm BestCostCycle}
%\bigskip
%
%To prove the correctness of the algorithm let \textit{O} be an optimal orientation 
%of $C_n$.
%If \textit{O} is a clockwise or counterclockwise cycle then it will be found in step 1.
%Otherwise, let \textit{i} be a vertex such that either both of the edges $\{i-1,i\}$ and $\{i,i+1\}$
%are oriented away from \textit{i} in \textit{O}, or both of them are oriented towards \textit{i}.
%In the former case, any optimal orientation of the path $P^{out}$ formed in 
%the $i$-the execution of the loop yields an
%orientation of $C_n$, because both edges involving $i$ are outward oriented
%by the construction of $P^{out}$ in step \ref{i.c1}. This orientation has the same cost 
%as \textit{O}, and it will be found in step \ref{i.c2}.
%Similarly, in the latter case, any optimal orientation of $P^{in}$ yields an optimal
%orientation of $C_n$, and it will be found in step \ref{i.c4}.
%
%Since a call to $BestCostPath(P)$ takes $O(n^2)$ time the algorithm runs in $O(n^3)$ time.

% bibliography, glossary and index would go here.

\bibliography{orientation}{}
\bibliographystyle{plain}



\end{document}