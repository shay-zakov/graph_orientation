\documentclass[a4paper,12pt]{article}
%\usepackage[utf8]{inputenc}
\usepackage{multirow}
\usepackage{amsmath}
\usepackage{amsthm}
\usepackage{graphicx}


% Definintg \devc for double-arrow vector notation:
\newcommand\shrinkage{2.1mu}
\newcommand\vecsign{\mathchar"017E}
\newcommand\dvecsign{\smash{\stackon[-1.95pt]{\mkern-\shrinkage\vecsign}{\rotatebox{180}{$\mkern-\shrinkage\vecsign$}}}}
\newcommand\dvec[1]{\def\useanchorwidth{T}\stackon[-4.2pt]{#1}{\,\dvecsign}}
\usepackage{stackengine}
\stackMath
%\usepackage{graphicx}

% Commands for annotating path directions - either entire path or just last edge
% TODO: workout the appropriate arrow styls
\newcommand{\er}[1]{\ensuremath{#1^\rightarrow}} % Last edge pointing right
\newcommand{\el}[1]{\ensuremath{#1^\leftarrow}} % Last edge pointing left
\newcommand{\elr}[1]{\ensuremath{#1^\leftrightarrow}} % Last edge pointing either right or left
\newcommand{\ar}[1]{\ensuremath{\vec{#1}}} % All edges pointing right
\newcommand{\al}[1]{\ensuremath{\cev{#1}}} % All edges pointing left
\newcommand{\alr}[1]{\ensuremath{\dvec{#1}}} % All edges pointing either right or left


\usepackage[linesnumbered,ruled,vlined,scleft]{algorithm2e}
%\DontPrintSumicolon % Some LaTeX compilers require you to use \dontprintsemicolon instead
\SetKw{Break}{break}
\SetKw{And}{and}
\SetKwComment{Precondition}{\textbf{precondition:~}}{}
\SetKwComment{Postcondition}{\textbf{postcondition:~}}{}
\SetKwComment{LoopInv}{\textbf{loop invariant:~}}{}
\SetKwComment{Condition}{\textbf{condition:~}}{}


\newtheorem{theorem}{Theorem}[section]
\newtheorem{corollary}{Corollary}[theorem]
\newtheorem{lemma}{Lemma}[section]
\newtheorem{observation}{Observation}[section]
\newtheorem{claim}{Claim}[section]
\newtheorem{property}{Property}[section]
\newtheorem{problem}{Problem}[section]
\newtheorem{definition}{Definition}[section]
\newtheorem{remark}{Remark}[section]


%\makeatletter
%\g@addto@macro\algorithm{\setcounter{cond}{0}}
%\makeatother   

\makeatletter
\DeclareRobustCommand{\cev}[1]{%
	{\mathpalette\do@cev{#1}}%
}
\newcommand{\do@cev}[2]{%
	\vbox{\offinterlineskip
		\sbox\z@{$\m@th#1 x$}%
		\ialign{##\cr
			\hidewidth\reflectbox{$\m@th#1\vec{}\mkern4mu$}\hidewidth\cr
			\noalign{\kern-\ht\z@}
			$\m@th#1#2$\cr
		}%
	}%
}
\DeclareMathOperator*{\argmin}{arg\,min}
\makeatother

\begin{document}
\sloppy
	

\author{}
\title{Orienting an Edge-Bi-Weighted Graph to Minimize the Heaviest Path}
\date{}


\newcommand{\vecP}{\vec{P}}
\newcommand{\vecO}{\vec{O}}
\newcommand{\vecG}{\vec{G}}
\newcommand{\dir}[1]{\stackrel{\rightarrow}{#1}}
\newcommand{\dirO}[1][O]{\stackrel{\rightarrow}{#1}}
%\newcommand{\rdir}[1]{\rho ({#1})}
%\newcommand{\ldir}[1]{\lambda ({#1})}
\newcommand{\rdir}[1]{\vec{#1}}
\newcommand{\ldir}[1]{\cev{#1}}


\maketitle
%\tableofcontents

\section{Introduction}

A \emph{edge bi-weighted graph} is an undirected graph $G = (V, E)$ in which each edge $\{u, v\} \in E$ has a pair of (possibly different) weights $w(u, v)$ and $w(v, u)$ associated with its two possible orientations $(u, v)$ and $(v, u)$, respectively. An \emph{orientation} 
of $G$ creates a directed graph $\vec{G}$ 
by selecting for each undirected edge $\{u, v\} \in E$ exactly one of its two possible orientations. Denote by $\mathcal{O}(G)$ the set of all orientations of $G$. 

The problem we address is the following.

\noindent {\bf Input}: An edge bi-weighted graph $G$.

\noindent {\bf Output}: An orientation  of $G$ that minimizes the weight of 
the heaviest of the resulting simple directed paths.

Actually, there remains an ambiguity in the specification of the cost function because it 
fails to specify what kinds of paths the minimum is to be taken over: all simple 
directed subpaths or only maximal ones. To distinguish between these two possibilities we define two measures.

\begin{definition}
	Given a directed path $\vec{P}=<v_0,\ldots,v_n>$ denote its cost
	\begin{equation}\label{eq.W}
		W(\vec{P})=\sum_{k=0}^{n-1} w(v_k,v_{k+1}).
	\end{equation}
	We define two cost measures for an oriented graph $\vec{G}$:
	\begin{align} 
		h_s(\vecG) &= \max\{W(\vec{P})\mid \vecP \mbox{ is a simple path in }\vecG \},\label{eq.hs} \\
		h_m(\vecG) &= \max\{W(\vec{P})\mid \vecP \mbox{ is a maximal simple path in }\vecG \}.\label{eq.hm}
	\end{align}
	The corresponding cost functions for orienting an undirected graph are 
	\begin{align}
		H_s(G) &= \min\{h_s(\vecG)\mid \vecG \in \mathcal{O}(G)\}, \\
		H_m(G) &= \min\{h_m(\vecG)\mid \vecG \in \mathcal{O}(G)\}.
	\end{align}
\end{definition}
Note that always $h_s(\vecG)\geq 0$, because the empty path is one of the possible simple paths appearing in equation (\ref{eq.hs}).

\noindent {\bf Example}: To illustrate the differences between the two cost functions consider
 $\vec{P}=<v_0,v_1,v_2,v_3>$ and $\vec{P_{1,2}}=<v_1,v_2>$, with $w(v_0,v_1)=2, w(v_1,v_2)=-3,w(v_2,v_3)=6$.
 Then $h_m(\vec{P})=W(\vec{P})=5, h_s(\vec{P})=6$, and 
 $h_m(\vec{P_{1,2}})=-3, h_s(\vec{P_{1,2}})=0$.
 \qed

A very useful property of $H_s$ is that it is monotone: the cost of a graph is never less than the cost of a subgraph.
\begin{lemma}\label{lem:sprop}
	Given any edge bi-weighted graph G,
	$H_{s}(G)\geq  H_{s}(G')$ for any subgraph $G'$ of $G$. 
\end{lemma}

\begin{proof}
It is clear from the definition (\ref{eq.hs}) of $h_s$ that $h_s(\vecP)\geq h_s(\vecP ')$for any subpath 
$\vecP '$ of $\vecP$.  
\end{proof}

This property sets $H_m$ apart from $H_s$,  as the above example shows:
$h_{m}(\vecP)=5 <  h_{m}(\vecP')=6$ for the subgraph $\vecP '=<v_2,v_3>$ of $\vecP$.

However, if all weights are non-negative $H_m$, too, is monotonic. In fact, in that case the two definitions coincide.

\begin{lemma}
	If all of the weights an edge-bi-weighted graph $G$ are non-negative 
	then $h_{s}( \vecG )= h_{m}( \vecG )$ for any orientation $\vecG$ of $G$.
	In particular,  $H_{s}(G)= H_{m}(G)$,  and 
	$\vecG$ is an optimal orientation of $G$ with respect to $h_{s}$
	if and only if it is optimal with respect to $ h_{m} $.
\end{lemma}
\begin{proof}
From the definitions of $H_m$ and $H_s$ it is clear that  $h_{m}(\vec{P})\leq  h_{s}(\vec{P})$
for any directed path $\vec{P}$, and that the inequality is in fact an equality if all the weights 
are non-negative. This implies that $h_{s}( \vecG )= h_{m}( \vecG )$ for any orientation $\vecG$ of $G$.
\end{proof}

{\bf Conjecture}:
\textit{Given an edge bi-weighted graph $G$, any orientation that is optimal with 
	respect to $h_{m}$ is also optimal with respect to $h_{s} $.}

\medskip
In the coming sections we will consider the following two problems
for various classes of graphs.

\begin{problem}
	Find an algorithm to compute
	$H_s(G) $, or an optimal orientation of $G$ under $H_s$,
	for any $G$ in a given class of edge bi-weighted graphs.
%	Given a class of edge bi-weighted graphs find an algorithm to compute
%	$H_s(G) $, or an optimal orientation of $G$,
%	for any $G$
%	in the class.
\end{problem}

\begin{problem}
Find an algorithm to compute
$H_m(G) $, or an optimal orientation of $G$ under $H_m$,
for any $G$ in a given class of edge bi-weighted graphs.
\end{problem}

\begin{table*}[h!]
	\begin{center}
		\caption{Summary of Results.}
		\label{tbl:summary}
		\begin{tabular}{|l|l|l|l|}
			\hline
			graph class & time under $H_s$ & time under $H_m$  & Theorems  \\
			\hline
			star            &        $O(n \log n)$  & $O(n \log n)$        & \ref{t.star}, Algorithm \ref{a.stars}		\\
						\hline
			linear graph & $O(n)$        		&$O(n \log n)$ 	& \ref{t.linear-s}, \ref{t.linear-m} \\
						\hline
			cycle 		 & $O(n)$ 		& $O(n^2 \log n)$ 		&  \ref{t.cycle-s}, \ref{t.cycle-m} \\
									\hline
			$k$-legged spider 		 & $O(k^22^k n^2)$ 		& $O(k^22^kn^2 \log n)$ 		&   \\
%			Theorem 9 & $2 \lceil \log n \rceil$ & $D/D^2/d$ & Encoder & BTN \\
%			Theorem 15 & $ \lceil \log n \rceil$ & 4 layers ($O(\sqrt{n}+D)$ & Encoder & BTN \\
%			\hline
%			Theorem 19 & $2 \lceil \sqrt{n} \rceil$ & $D/({\frac d 2}+D)/d/{\frac {dD} {2}}/D$
%			& Encoder/Decoder & BTN \\
%			Theorem 21 & $\lceil \log n \rceil$ &  $D/n/d/n/D$ & Encoder/Decoder & BTN \\
%			Theorem 22 & 
%%			$\textcolor{red}
%			${2\lceil \log \sqrt{n} \rceil}$ & 7 layers ($O(D \sqrt{n})$ nodes) & Encoder/Decoder & BTN \\
			\hline
		\end{tabular}
	\end{center}
\end{table*}

\section{Algorithms for linear graphs \label{s.2}}
We call a graph that has two vertices of degree 1 and all other vertices of degree 2  a \emph{ linear graph}, and 
reserve the term \emph{path} for a directed linear graph that has a single source (and a single sink).
Given a linear graph  $L$ on $n+1$ vertices number its vertices from \textit{0} to \textit{n}, and denote the weights of
edges $(i,i+1)$ and  $(i+1,i)$ by $w(i,i+1)$ and $w(i+1,1)$, respectively.

We describe next a generic algorithm for finding the cost of an optimal orientation 
of a bi-weighted linear graph. This high level version makes no use of the details
of the cost function, be it $H_m$ or $H_s$. Its description therefore uses the subscript $x$ with $x\in \{s,m\}$. However, the implementations under 
the two cost functions of this high level version, presented in the succeeding subsections, reveal surprising differences between their running times.

\subsection{A generic algorithm for linear graphs}

%\bigskip

{\bf Notation}:
\begin{itemize}
\item $L_{i, j}$ is the sub-graph of $L$ induced by the vertices $i,  \ldots, j$. 
\item $\vec{L}_{i, j}$ denotes the oriented version 
of $L_{i, j}$ in which all edges are directed to the right (i.e. of the form $(i, {i+1})$),
and $\cev{L}_{i, j}$ denotes the oriented version 
of $L_{i, j}$ in which all edges are directed to the left. 
\item $\rlast{H}_x(i)$ is the value of an optimal orientaton of $L_{i,n}$ under the constraint
that edge $\{i,i+1\}$ is directed towards $i+1$, and $H_x^{\prec}(i)$ is the value of an optimal orientaton of $L_{i,n}$ under the constraint
that edge $\{i,i+1\}$ is directed towards $i$.\\
In particular, the cost of an optimal orientation of $L$ under $H_x$ is 
$\min \{H_x^{\succ}(0),\ H_x^{\prec}(0)\}$.
\end{itemize}

The basic step of the recursive algorithm for finding an optimal orientation, or its cost, is to locate the last 
vertex of an optimal orientation of $L_{ j,n}$ at which there is a change in direction,
given that the very last edge has a specified direction. Keeping this in mind the proof 
of correctness of the algorithm is straightforward. 
For simplicity we outline the algorithm for finding the cost of an optimal orientation, BestCostLinear$_x(L)$.
That algorithm is easily converted into an algorithm for finding the orientation itself, BestOrientLinear$_x(L)$,
 by recording for each 
$H^{\succ}(j)$ and $H^{\prec}(j)$ which $i$ corresponds to the minimum in statements 
\ref{st.1} and \ref{st.2}.
%\bigskip

\begin{algorithm}
	\KwIn{a bi-weighted linear graph $L$ on $n+1$ vertices}
\KwOut{an optimal orientation of $L$ under $H_x$}
	$H^{\succ}(n)=H^{\prec}(n)=-\infty$, $n=length(L)+1$\;
	\For{$i=n-1$ to $0$}{
		set $H^{\succ}(i)$ to the minimum over $i< j \leq n$ of $\max \{h_x(\vec{L}_{i, j}), H^{\prec}(j)\}$\;
		\label{st.1}
		set $H^{\prec}(j)$ to the minimum over $ i <j \leq n$ of  $\max \{h_x(\cev{L}_{i, j}), H^{\succ}(i)\}$\;
			\label{st.2}
	}
	\Return{$\min \{H^{\succ}(0),\ H^{\prec}(0)\}$}\;
	\caption{BestCostLinear$_x(L)$}
	\label{algo:H}
\end{algorithm}
\begin{theorem}
Given input $L$, Algorithm \emph{BestCostLinear}$_x$ finds the cost of an optimal orientation of $L$.
\end{theorem}
A straightforward implementation of this algorithm runs in $O(n^2)$ time. 
The implementations described in the coming two subsections provide drastic improvements.

\subsection{Running time under cost function $H_s$}
In this subsection we show that 
Algorithm BestCostLinear$_s$ can be made to run in time $O(n)$ by carefully implementing
the computational operations of the algorithm.
Two issues have to be addressed.

	The first is the time needed to compute $h_s(\vec{L}_{i, j})$ 
	and $h_s(\cev{L}_{i, j})$ for given $0\leq i<j\leq n$.
	Focusing on $h_s(\vec{L}_{i, j})$, and combining equations (\ref{eq.W}) and (\ref{eq.hs}):
	\begin{equation}\label{eq.hsij}
	h_s(\vec{L}_{i, j})=\max \left\{ \sum_{t=i'}^{j'-1}w(t,t+1) \mid i\leq i' \leq j' \leq j\right\}.
	\end{equation}
	
	Computing $h_s(\vec{L}_{i, j})$ for a pair $(i,j)$ is therefore an instance
	of the Range Maximum-sum Segment On-line Query problem, 
	RMSOQ for short, as defined in \cite{chen2007range}:
	\begin{problem}[Range Maximum-sum Segment On-line Query problem]\ \\
		\noindent \emph{\bf Input to be preprocessed:}
		A nonempty sequence $a_1 ,\ldots a_n$ of real numbers.\\ 
		\noindent \emph{\bf Online query:} respond to a query of the form $RMSOQ( i, j)$ by returning a pair of indices $(i', j')$ 
		which maximize
		$\sum_{t=i'}^{j'}a_t$ over all $i\leq i' \leq j' \leq j$.	
	\end{problem}
	Chen and Chao \cite{chen2007range} presented a method for answering each such query 
	in constant time after  $A$ is preprocessed in $O(n)$ time. The following Lemma 
	summarizes the discussion and its relevance.
\begin{lemma}
	Suppose $w(i,i+1),0\leq i <n$ and $w(i+1,i),0\leq i <n$ have been preprocessed 
	in linear time for the RMSOQ problem.
	After  $H^{\prec}(i)$ and $H^{\succ}(i)$ have been computed for $0\leq i <j$ each 
	value of the form $\max \{ H^{\prec}(i), h_s(\vec{L}_{i, j})\}$ and $\max \{ H^{\succ}(i), h_s(\cev{L}_{i, j})\}$ appearing in statements \ref{st.1} and \ref{st.2} in iteration $j$
	can be evaluated in constant time.
\end{lemma}
A question remains: how many queries $RMSOQ( i, j)$ will the algorithm present?
The answer to this question will be found by addressing the second issue: what is the time needed to find all minimum values in statements \ref{st.1} and \ref{st.2}. Here the notion of a totally monotone matrix will be helpful. 
To emphasize that this monotinicity is rowwise we call it r-monitinicity.
\begin{definition}\label{d.tm}
	Given an $n \times n$ matrix $M$, denote
	by $C(i)$ the least column index at which the minimum in row $i$ of $M$ is achieved, 
	i.e.,
	$C(i) = \min \{k :M_{i,k} = min_{1\leq j \leq n} M_{i,j}\}$.
	The matrix $M$ is r-monotone if $C(1) \leq  C(2)\leq \cdots \leq C(n)$, and it is totally c-monotone 
	if all $2\times 2$
	submatrices of $M$ formed by choosing two rows and two columns are r-monotone.
\end{definition}
The following Lemma is easily verified.
\begin{lemma}\label{l.rmono}
	$M$ is totally r-monotone if and only if $M(i_1,j_1)\leq M(i_1,j_2)$
	whenever $M(i_2,j_1)\leq M(i_2,j_2)$,
for all $i_1<i_2$ and $j_1<j_2$.
\end{lemma}
Whereas for an arbitrary square matrix it takes quadratic time to 
compute all values $R(j)$, for a c-monotone matrix the SMAWK algorithm \cite{smawk1987} 
is able to do so, off-line, in linear time.
	
Define the matrices $M^{\succ}(i,j)$ and $M^{\prec}(i,j)$ as follows.
\begin{itemize}
	\item For $0\leq i<j \leq n$,
$$M^{\succ}(i,j)=\max \{h_s(\vec{L}_{i, j}), H^{\prec}(j)\},\ 
M^{\prec}(i,j)=\max \{h_s(\cev{L}_{i, j}), H^{\succ}(i)\}.$$	
	\item $M^{\succ}(i,j)=M^{\prec}(i,j)=\infty$  for $1\leq j \leq i \leq n $.
%	\item $M^{\succ}(0,j)=h_s(\vec{L}_{0, j})$, $M^{\prec}(0,j)=h_s(\cev{L}_{0, j})$, $1\leq j\leq n$.
\end{itemize}
In these terms the algorithm computes the minimum value in row $i$ of 
the matrices $M^{\succ}$
and $M^{\prec}$, for $0\leq i <n$.
Two features of this computation deserve particular attention.
\begin{enumerate}
	\item The computation of $M^{\succ}$
	and $M^{\prec}$ has an on-line flavor: before computing the minimum value in row $i$ of 
	the matrix $M^{\succ}$, or $M^{\prec}$, the minimum values of all rows $i<i'$ of 
	the matrix $M^{\prec}$, respectively $M^{\succ}$,  have to be available.
		\item \label{i.1} It follows from equation (\ref{eq.hsij}) that
	$h_s(\vec{L}_{i, j})$ and $h_s(\cev{L}_{i, j})$ are both non-increasing in $i$ 
	and non-decreasing in $j$. 
\end{enumerate}
According to the first feature, the problem of computing 
$H^{\succ}(j)$ and $H^{\succ}(j)$ can be dealt with by any method that solves the 
 following problem
\begin{problem}[ORM - Online Row Minima]
	For $1\leq i \leq n$ compute $H(i)=\min \{M(i,j) \mid 1\leq j \leq n\}$, where 
	the values of $H(i'),\ i< i'\leq n$ have to be computed before $M(i,j)$ can be evaluated.
\end{problem}
In our case, both $M^{\prec}$ and $M^{\succ}$ are of the form $M(i,j)=\max \{f(i,j),g(j)\}$,
where $f$ is non-increasing in $i$ and non-decreasing in $j$, according to the 
above second feature. This is the key to proving that $M^{\prec}$ and $M^{\succ}$ are totally r-monotone.
\begin{proposition}
	If $f$ is non-increasing in $i$ and non-decreasing in $j$ and $M(i,j)=\max \{f(i,j),g(j)\}$,
	then $M$ is totally r-monotone.
\end{proposition}
\begin{proof}
	According to Lemma \ref{l.rmono} we have to prove that for any $i_1<i_2$ and $j_1<j_2$,
	 $M(i_2,j_1)\leq M(i_2,j_1)$ implies $M(i_1,j_1)\leq M(i_1,j_2)$.

We claim that, for any $i$ and  $j_1<j_2$, 
\begin{equation}\label{e.iff}
M(i,j_1)\leq M(i,j_2) \mbox{ if and only if }  g(j_1)\leq \max \{ f(i, j_2),g(j_2)\}. 
\end{equation}
This follows from the definition of $M$, and the fact that 
 $f(i, j_1)\leq f(i, j_2)\leq \max \{ f(i, j_2),g(j_2)\} $ for $j_1 < j_2$.

Consequently,  if $M(i_2,j_1)\leq M(i_2,j_2)$ then
$$ g(j_1)\leq \max \{ f(i_2,j_2), g( j_2)\}\leq \max \{ f(i_1,j_2), g( j_2)\},$$
since $ f(i_2, j_2) \leq f(i_1, j_2)$.
Using equation (\ref{e.iff}) again, $M(i_1,j_1)\leq M(i_1,j_2)$.
\end{proof}
Summarizing the foregoing discussion, to find all minimum values in statements \ref{st.1} and \ref{st.2} of the algorithm we can employ the solution to the following problem.
\begin{problem}[ORMM - Online Row Minima of  a r-Monotone matrix]\label{p.ormm}
		For $1\leq j \leq n$ compute $H(j)=\min \{M(i,j) \mid 1\leq j \leq n\}$, where 
		$M$ is r-monotone and
	the values of $H(i'),\ i < i' \leq n$ have to be computed before $M(i,j)$ can be evaluated.
\end{problem}
Note that the SMAWK algorithm cannot
be used directly because it is off-line whereas our computation has to be online.
Fortunately, several linear-time online algorithms for solving Problem \ref{p.ormm} have been published, 
\cite{klawe89,larmore91,galil92,barnoy09}.

\begin{theorem}\label{t.linear-s}
	When the cost function used in Algorithm \emph{BestCostLinear}$_x$ is $H_s$ its running time is $O(n)$.
\end{theorem}

\

\subsection{Min-Max Variants}

Let $f$ and $g$ denote two functions that get as an input a single integer $i \geq 0$. The \emph{Min-Max} function over $f$, $g$, and an integer $j$, denoted by \emph{MM}, is defined as follows:

\begin{align}
	\text{MM}(f, g, j) &= \min_{0 \leq i < j} \max \left\{f(i), g(i)\right\}
	\label{eq:MM} %\\
%	\text{ARG-MM}(f, g, j) &= \argmin_{0 \leq i < j} \max \left\{f(i), g(i)\right\}.
%	\label{eq:ARG-MM}
\end{align}


In this section we consider the problem of computing MM$(f, g_j, j)$ 
%and ARG-MM$(f, g_j, j)$ 
for a fixed function $f$ and a series of functions $g_1, g_2, \ldots, g_n$, where the latter series adheres to two different sets of restrictions. These restrictions are defined as follows:

\begin{definition}
	\label{def:type1}
	A series of functions $g_1, g_2, \ldots, g_n$ is of \emph{Type 1} if every function $g_j$ in the series is monotonously non-increasing, and for every two consecutive functions $g_{j-1}$ and $g_j$ and for every integer $0 \leq i < j-1$, $g_j(i) \geq g_{j-1}(i)$.
\end{definition}

\begin{definition}
	\label{def:type2}
	A series of functions $g_1, g_2, \ldots, g_n$ is of \emph{Type 2} if for every $j$ there exists some constant $c_j$ such that for every $0 \leq i < j-1$, $g_j(i) = g_{j-1}(i) + c_j$, and $g_j(j-1) = c_j$. 
\end{definition}

[TODO: find better terms than type 1 and 2.]
[TODO: explain how type 1 series emerges from the longest paths and type 2 series emerges from longest maximal paths.]

\subsubsection{Type 1}
\label{sec:type1}

In the current section, assume a function $f$ and a Type 1 function series $g_1, g_2, \ldots, g_n$ are given.

\begin{definition}
\label{def:dominant}
	An index $0 \leq i < j$ is \emph{dominant} with respect to $j$ if for every $0 \leq i' < i$ it holds that $\max\left\{f(i), g_j(i)\right\} \leq \max\left\{f(i'), g_j(i')\right\}$. Denote by $D_j$ the set of all dominant indices with respect to $j$, and by  $\optIx$ the maximal index in $D_j$. 
\end{definition}

\begin{claim}
\label{clm:dominant}
	For every $0 < j \leq n$, $\text{MM}(f, g_j, j) = \max\left\{f(\optIx), g_j(\optIx)\right\}$.
\end{claim}

\begin{proof}
	By definition $\max\left\{f(\optIx), g_j(\optIx)\right\} \leq \max\left\{f(i), g_j(i)\right\}$ for every $0 \leq i < \optIx$, therefore to prove the claim it remains to show that $\max\left\{f(\optIx), g_j(\optIx)\right\} \leq \max\left\{f(i), g_j(i)\right\}$ for every $\optIx < i < j$. Let $i'$ be an index that minimizes $\max\left\{f(i), g_j(i)\right\}$ for every $\optIx < i < j$. Since $i' > \optIx$ and $\optIx$ is the maximal dominant index with respect to $j$, $i'$ is not dominant with respect to $j$. In particular, there is some index $0 \leq i'' < i'$ such that $\max\left\{f(i''), g_j(i'')\right\} < \max\left\{f(i'), g_j(i')\right\}$. From the selection of $i'$ it cannot be that $\optIx < i'' < i'$, and so $0 \leq i'' \leq \optIx$. Since $\optIx$ is dominant with respect to $j$, $\max\left\{f(\optIx), g_j(\optIx)\right\} \leq \max\left\{f(i''), g_j(i'')\right\} < \max\left\{f(i'), g_j(i')\right\} \leq \max\left\{f(i), g_j(i)\right\}$ for every $\optIx < i < j$, as required.
\end{proof}

We next show how can indices of the form $\optIx$ can be efficiently computed, given the preceding indices $\optIx[j-1]$.

\begin{claim}
	\label{clm:dominance_a}
	$i \in D_j$ if and only if $f(i) \leq \max\left\{f(i'), g_j(i')\right\}$ for every $0 \leq i' < i$.
\end{claim}

\begin{proof}
	Let $i$ and $i'$ be indices such that $0 \leq i' < i$.
	Since $g_j$ is monotonously non-increasing, $g_j(i) \leq g_j(i') \leq \max\left\{f(i'), g_j(i')\right\}$. Therefore, $\max\left\{f(i), g_j(i)\right\} \leq \max\left\{f(i'), g_j(i')\right\}$ if and only if $f(i) \leq \max\left\{f(i'), g_j(i')\right\}$. By the definition of dominance, $i \in D_j$ if and only if the latter inequality holds for every $0 \leq i' < i$.
\end{proof}

\begin{claim}
	\label{clm:dominance_b}
	If $i \in D_{j-1}$ then $i \in D_j$.
\end{claim}

\begin{proof}
	Let $i$ be an index such that $i \in D_{j-1}$. Due to Claim~\ref{clm:dominance_a}, to prove the claim we only need to show that $f(i) \leq \max\left\{f(i'), g_j(i')\right\}$ for every $0 \leq i' < i$.
	Let $i'$ be any such index.
	If $f(i) \leq f(i')$, it immediately follows that $f(i) \leq \max\left\{f(i'), g_j(i')\right\}$.
	Otherwise  $f(i) > f(i')$. Since $i \in D_{j-1}$ we get that $f(i') < f(i) \leq \max\left\{f(i'), g_{j-1}(i)\right\}$, and so $\max\left\{f(i'), g_{j-1}(i)\right\} = g_{j-1}(i)$. Thus, $f(i) \leq g_{j-1}(i') \stackrel{\text{Type 1}}{\leq} g_j(i') \leq \max\left\{f(i'), g_j(i')\right\}$, as required.
\end{proof}


Note that if $i \in D_j$ and $i = j-1$, it must be that $i = \optIx$. Otherwise, Claim~\ref{clm:dominance_next} below describes a property that allows, given \emph{some} dominant index $i$ with respect to $j$, to traverse an increasing sequence of dominant indices in order to identify $\optIx$ without considering non-dominant indices. More importantly, it provides a condition that allows to identify $\optIx$ once it has been reached.

\begin{claim}
	\label{clm:dominance_next}
	Let $i < j-1$ be an index such that $i \in D_j$, and let $k = \argmin\limits_{i < i' < j} f(i')$. 
	If $f(k) > \max\left\{f(i), g_j(i)\right\}$ then $i = \optIx$, and otherwise $k \in D_j$.
\end{claim}

\begin{proof}
	For the case when $f(k) > \max\left\{f(i), g_j(i)\right\}$, it follows from the selection of $k$ that for every $i < i' < j$ it holds that $\max\left\{f(i), g_j(i)\right\} < f(k) \leq f(i') \leq \max\left\{f(i'), g_j(i')\right\}$, and so $i = \optIx$.
	
	For the case when $f(k) \leq \max\left\{f(i), g_j(i)\right\}$, we need to show that $k \in D_j$. From Claim~\ref{clm:dominance_a}, it is sufficient to show that $f(k) \leq \max\left\{f(i'), g_j(i')\right\}$ for every $0 \leq i' < k$. For $i' \leq i$ this follows immediately from the fact that $i$ is dominant respect to $j$. For $i < i' < k$, the inequality holds since $f(k) \leq f(i') \leq \max\left\{f(i'), g_j(i')\right\}$.
\end{proof}

In order to utilize Claim~\ref{clm:dominance_next} for the efficient identification of $\optIx$, there is also a need to be able to efficiently find an index $k$ such that $k = \argmin\limits_{i < i' < j} f(i')$. This is known as the Range Minimum Query problem [TODO: add citations], for which there are efficient implementations that allow, after linear preprocessing time, to execute each query in a constant time. In our case, as will be shown later, the function $f$ is not available in its entirety for preprocessing - values of the form $f(j)$ will be sequentially computed throughout the algorithm's run, while range minimum queries over $f$ will have to be performed prior to the full computation of $f$. Nevertheless the series of queries executed throughout the computation has the property of having monotonously increasing endpoints (that is, for every pair of endpoints $(i, j)$ and $(i', j')$ in two consecutive queries, it holds that $i \leq i'$ and $j \leq j'$). This is known as the \emph{Sliding RMQ} problem, for which there is a significantly simpler solution than those for general RMQ~\cite{lee2007simple}. Besides its simplicity, this solution does not require the entire array over which the queries are performed to be given prior to the series of queries. It maintains a data structure that allows the sequential addition of elements at the end of the array, and executing queries given that their endpoints are within the current array length. In this paper, we will denote such a structure by the letter $R$. The procedure ArgMin$(R, i, j)$ returns an index $i < k < j$ that minimizes $f(k)$ when $i < j-1$, or NULL when $i \geq j-1$. We will write $R \sim R_{i, j}$ to indicate the internal state of $R$ allows to execute a query ArgMin$(R, i', j')$ such that $i' \geq i$ and $j' \geq j$, provided that $f(i'+1), f(i'+2), \ldots, f(j'-1)$ are available to the ArgMin procedure.
While some queries might take more than a constant time for their execution, the overall running time for executing $m$ queries over an array whose final length is $n$, is $O(m+n)$. 
[TODO: consider adding an appendix with implementation details.] 
Algorithm~\ref{algo:maxDominant} computes $\optIx$, given some $i \in D_j$ (later we use $\optIx[j-1]$ as the initial index when applying this procedure) and the data structure $R$.

%\begin{algorithm}
%	\Precondition{$i$ is dominant with respect to $j$, and $D ~ D_{i, j}$ for some $i' \leq i$ and $j' \leq j$}
%	\If{$i < j-1$}{
%		$k \gets \text{ArgMin}(D, i, j)$\;
%		\LoopInv{$i < j-1$, $i$ is dominant with respect to $j$, $k = \argmin_{i < i' < j} f(i')$, and $D = D_{i, j}$ }
%		\While{$f(k) \leq \max\left\{f(i), g_j(i)\right\}$}{
%			\Condition{$k$ is dominant with respect to $j$ and $k > i$}
%			$i \gets k$\;
%			\If{$i = j-1$}{\Break}
%			$k \gets \text{ArgMin}(D, i, j)$\;
%		}
%	}
%	\Postcondition{$i = \optIx$}
%	\Return{$i$}\;
%	\caption{MaxDominant$(f, g_j, D, i, j)$}
%	\label{algo:maxDominant}
%\end{algorithm}

\begin{algorithm}
	\Precondition{$i \in D_j$, and $R \sim R_{i, j}$}
	$k \gets \text{ArgMin}(R, i, j)$\;
	\LoopInv{$i \in D_j$, $k = \argmin\limits_{i < i' < j} f(i')$ (or NULL if $i \geq j-1$), and $R \sim R_{i, j}$ }
	\While{$i < j-1$ \And $f(k) \leq \max\left\{f(i), g_j(i)\right\}$}{
		\Condition{$k \in D_j$ and $k > i$}
		$i \gets k$\;
		$k \gets \text{ArgMin}(R, i, j)$\;
	}
	\Postcondition{$i = \optIx$, and $R \sim R_{\optIx, j}$}
	\Return{$i$}\;
	\caption{MaxDominant$(f, g_j, R, i, j)$}
	\label{algo:maxDominant}
\end{algorithm}


\paragraph{Time complexity.}
Consider the sequential execution of Algorithm~\ref{algo:maxDominant} in order to compute $\optIx$ for $j = 1, 2, \ldots, n$. For each such index $j$, $\optIx$ is set to be the result of MaxDominant$(f, g_j, R^{j-1}, \optIx[j-1], j)$, where $R^{j-1}$ denotes the state of $R$ at the end of the previous iteration, with the initial settings $\optIx[0] = 0$ and $R^0 = R_{0, 0}$. It can be noticed that the condition $R^{j-1} \sim R_{\optIx[j-1], j}$ is maintained throughout the run. Also, it can be seen that the running time of each call to the procedure is dictated by the running time of its internal calls to the ArgMin procedure (as there are $O(1)$ additional operations per call to ArgMin). In each call to ArgMin either the index $i$ or the index $j$ used in the query strictly increases with respect to the previous call, and therefore there are overall at most $2n$ calls to ArgMin throughout the run. As shown in~\cite{lee2007simple}, the overall running time of all ArgMin queries is $O(n)$.


\subsubsection{Type 2}
\label{sec:type2}
In this section we assume a function $f$ and a Type 2 function series $g_1, g_2, \ldots, g_n$ are given. For convenience, we define an auxiliary function $g$, where $g(i) \equiv g_i(0)$ for $i > 0$ and $g(0) = 0$. We also assume that querying the minimum or maximum values in an empty set of values yield $\infty$ or $-\infty$ values, respectively.

\begin{claim}
\label{clm:type_2}
	For every $0 \leq i < j \leq n$, $g_j(i) = g(j) - g(i)$.
\end{claim} 

\begin{proof}
	For $j=1$ and $i=0$, we have by definition that $g_1(0) = g_1(0) - 0 = g(1) - g(0)$. 
	Assuming inductively the claim holds for every $j' < j$, we show it holds for $j$ too. Let $0 \leq i < j$. Then,
	$g_j(i) \stackrel{\text{Type 2}}{=} g_{j-1}(i) + c_j \stackrel{\text{induction}}{=} g(j-1) - g(i) + c_j \stackrel{\text{Type 2}}{=} g(j) - g(i)$.
\end{proof}

Consider the expression $\max\left\{f(i), g_j(i)\right\} \stackrel{\text{Claim~\ref{clm:type_2}}}{=} \max\left\{f(i), g(j) - g(i)\right\}$ for some $0 \leq i < j$. Clearly, $\max\left\{f(i), g_j(i)\right\} = f(i)$ when $g(j) \leq f(i) + g(i)$ and $\max\left\{f(i), g_j(i)\right\} = g_j(i) = g(j) - g(i)$ when $g(j) \geq f(i) + g(i)$. Denote $f^j = \hspace{-.5em}\min\limits_{{\tiny\begin{array}{c}
		0 \leq i < j, \\
		g(j) \leq f(i) + g(i)
		\end{array}}} \hspace{-1em} f(i)$
and $g^j = \hspace{-2em}\max\limits_{{\tiny\begin{array}{c}
		0 \leq i < j, \\
		g(j) \geq f(i) + g(i)
		\end{array}}} \hspace{-1em} g(i)$. Equation~\ref{eq:MM} can now be rewritten for the Type 2 scenario as follows:

\begin{equation}
\label{eq:MM2} 
	\begin{array}{rcl}
		\text{MM}(f, g_j, j) & = & \min\left\{
			\hspace{-.5em}\min\limits_{{\tiny\begin{array}{c}
				0 \leq i < j, \\
				g(j) \leq f(i) + g(i)
			\end{array}}} \hspace{-1em} \max\{f(i), g_j(i)\},
			\min\limits_{{\tiny \begin{array}{c}
				0 \leq i < j, \\
				g(j) \geq f(i) + g(i)
			\end{array}}} \hspace{-1em} \max\{f(i), g_j(i)\}
		\right\} \\
		& = & \min\left\{
			\hspace{-.5em}\min\limits_{{\tiny \begin{array}{c}
				0 \leq i < j, \\
				g(j) \leq f(i) + g(i)
				\end{array}}} \hspace{-1em} f(i), \ 
			g(j) - \hspace{-2em}\max\limits_{{\tiny\begin{array}{c}
				0 \leq i < j, \\
				g(j) \geq f(i) + g(i)
				\end{array}}} \hspace{-1em} g(i)
		\right\} \\
		& = & \min\left\{f^j, g(j) - g^j\right\}.
	\end{array}
\end{equation}

In order to compute the values $f^j$ and $g^j$ appearing at right-hand side of Equation~\ref{eq:MM2}, we next describe two data structures $F$ and $G$, that given $g(j)$ will allow the efficient retrievals of these values.
These two data structures will be implemented using \emph{sorted sets}.

A sorted set $S$ is a set of elements form $(k, v)$, where $k$ and $v$ are respectively referred to as the \emph{key} and \emph{value} of an element. We assume here the keys of $S$ are unique, and there is a total order "$\leq$" which is defined over the domain of keys.
In addition, the following operations are supported: 
\begin{enumerate}
	\item Find$(S, x)$ / FindPrev$(S, x)$ return an element $(k, v)$ in $S$ such that $x \leq k$ / $x > k$  and $k$ is minimal / maximal with respect to the order over the keys (or NULL if no such element exists), respectively;
%	\item FindPrev$(S, y)$ returns an element $s = (x, v)$ in $S$ such that $y > x$ and $x$ is maximal with respect to the order over the keys; % or NULL if $x \leq x'$ for every key $x'$ in $S$.
	\item Insert$(S, k, v)$ inserts an element $(k, v)$ into $S$, replacing if necessary an element $(k, v')$ with identical key if such an element exists in $S$;
	\item Min$(S)$ returns the minimal key in $S$.
\end{enumerate}

A standard efficient implementation for such a sorted set would be by using RB-trees [TODO: citations]. Such a data structure supports all of the above operations in $O(\log |S|)$ running time.

The $F$ sorted set is designed to compute values of the form $f^j$.
Given an index $j$, we will say that \emph{$F$ is valid for $x$} if the result of Find$(F, x)$ is a pair $(k, v)$ such that $v = \hspace{-.5em}\min\limits_{{\tiny \begin{array}{c}
		0 \leq i < j, \\
		x \leq f(i) + g(i)
		\end{array}}} \hspace{-1em} f(i)$.
The following two invariants will be used to describe the state of $F$ during the algorithm's run:

\paragraph{The $j$-invariant for $F$:}
$|F| = O(j)$, and $F$ is valid for any number $x$.

\paragraph{The relaxed $j$-invariant for $F$ and $(k, v)$:}
$|F| = O(j)$, and $F$ is valid for any number $x$ such that $k \leq x$ or Find$(F, x) = (k_x, v_x)$ and $v_x < v$.
\bigskip

[TODO: write the symmetric text to describe $G$.]

Initializing the sets $F$ and $G$ so they would admit the $0$-invariant is simply creating a set $F$ with the single element $(\infty, \infty)$ and a set $G$ with the single element $(-\infty, -\infty)$.
Let $F$ and $G$ admit the $j$-invariant for some $j > 0$. 
Consider the values $(k_F, v_f)$ and $(k_G, v_G)$ returned from executing Find$(F, g(j))$ and FindPrev$(G, g(j))$. From the $j$-invariant,
$$
v_F = \hspace{-1em}\min\limits_{{\tiny \begin{array}{c}
		0 \leq i < j, \\
		g(j) \leq f(i) + g(i)
		\end{array}}} \hspace{-1em} f(i) = f^j, \ 
v_G = \hspace{-1em}\max\limits_{{\tiny \begin{array}{c}
		0 \leq i < j, \\
		g(j) \geq f(i) + g(i),
		\end{array}}} \hspace{-1em} g(i) = g^j
$$ 
and so from Equation~\ref{eq:MM2} we have that $\text{MM}(f, g_j, j) = \min\{v_F, g(j) - v_G\}$.



In order to update $F$ and $G$ so they would sustain the $(j+1)$-invariant we do the following. For $F$, denote by $F^j$ the set which sustains the $j$ invariant, from which we show how to construct a set $F^{j+1}$ that sustains the $(j+1)$ invariant. In the latter set we need to consider a new possible value $f(j)$, which might be the value in some element returned due to a query of the form Find$(F^{j+1}, x)$. By definition, for $f(j)$ to be the returned element value, it must be that $f(j) + g(j) \geq x$. In particular, for every $y$ such that $f(j) + g(j) < y$, $f(j)$ should not be considered, and so the returned values from Find$(F^{j+1}, y)$ and Find$(F^{j}, y)$ should be te same. In addition, if Find$(F^{j}, x) = (k, v)$ such that $v < f(j)$, then clearly Find$(F^{j+1}, x)$ should also return an element whose value is $v$. Thus, $F^j$ admits the relaxed $(j+1)$-invariant for $(f(j) + g(j), f(j))$.

The following algorithm shows how given a data structure $F$ that maintains the $j$-invariant, it is updated to maintain the $(j+1)$-invariant.

\begin{algorithm}
	\Precondition{$F$ maintains the $j$-invariant}
	%	\tcp{Updating the data structures:}
	denote $x = f(j) + g(j)$, and let $(k, v) \gets \text{Find}(F, x)$\;
	\Condition{$F$ maintains the relaxed $(j+1)$-invariant for $(x^+, f(j))$}
	\If{$f(j) \leq v$}{
		Insert$(F, x, f(j))$\;
		\Condition{$F$ maintains the relaxed $(j+1)$-invariant for $(x, f(j))$}
		\If{$x > \text{Min}(S)$}{
			$(k, v) \gets \text{FindPrev}(F, x)$\;
			\LoopInv{$F$ maintains the relaxed $(j+1)$-invariant for $(k^+, f(j))$}
			\While{$f(j) \leq v$}{
				Remove$(F, k)$\;
				\Condition{$F$ maintains the relaxed $(j+1)$-invariant for $(k, f(j))$}
				\lIf{$x > \text{Min}(S)$}{$(k, v) \gets \text{FindPrev}(F, x)$}
				\lElse{\Break}
			}
		}
	}
	\Postcondition{$F$ maintains the $(j+1)$-invariant}
	\Return{$F$}\;
	\caption{Update-F$(F, f(j), g(j))$}
	\label{algo:MM2}
\end{algorithm}



\begin{algorithm}
	\Precondition{$S$ sustains the $S$-invariant for $j-1$}
%	\tcp{Updating the data structures:}
	set $x \gets f(j-1) + g_j(0)$\;
	let $s' \gets \text{Find}(S, x)$\;
	if $s'$ is not NULL, it is of the form $s' = (x', (k', F', G'))$. For the case where $s'$ is NULL, define $F'$
	\If{$s' = \text{NULL}$}{
		set $F \gets \min\left\{f(j-1), F'\right\}$ if $i < j-1$, otherwise $F_{j-1} \gets f(j-1)$\; 
		set $G_{j-1} \gets \max\left\{g_j(j-1), G_{K_i-1}\right\}$ if $i > 0$, otherwise $G_{j-1} \gets g_j(j-1)$\; 
	}
	\Else{
		set $F \gets \min\left\{f(j-1), F'\right\}$\; 
		set $G \gets \max\left\{g_j(0), G'\right\}$\; 
	}
	Insert$(S, x, (j-1, F, G))$\; 
	\Condition{$S$ sustains the $S$-invariant for $j$}
	\Return{$\min\{F, g_j(0) - G\}$}\;
	\caption{MM2$(f^d[\cdot], g^d[\cdot], d[\cdot], g_{j-1}(0), c_j, j)$}
	\label{algo:MM2}
\end{algorithm}

\section{Algorithms for cycles}\label{s.c}
Given a cycle graph $C$ on $n$ vertices, number its vertices from \textit{0} to \textit{n-1}, 
clockwise from an arbitrary node. 
Denote the weight of a directed
edge $(i,i+1)$ by $w(i,i+1)$. When a node $j$ with $j\geq n$ is referred to it 
should be understood as referring to node $j \mod n$. 
For example, $w(n-1,n)=w(n-1,0)$.

\subsection{An algorithm for cost function $H_s$.}
%If only the cost of an optimal orientation is of interest then the
%algorithm is deceptively short and sweet.
%\begin{algorithm}
%		let \textit{OneWayCost-s} be the least cost of a one-way orientation of the cycle\;
%	create a linear graph $L$ of 
%	length $3n$ by unrolling the cycle three times starting from vertex $0$;
%	number its vertices \textit{0} to \textit{3n}; and
%	equip edge $\{i,i+1\}$ of $L$ with the same weights as edge 
%	$\{i ,i+1\}$ of $C$\;
%	let $\vec{L}^*$ be the oriented graph returned by BestCostLinear-$s(L)$, and $h_s(\vec{L}^*)$ its cost\;
%	\lIf{$h_s(\vec{L}^*) \geq$ \textit{OneWayCost-s}}
%	{\Return \textit{OneWayCost-s}}
%	\If{$n$ is odd and every two consecutive edges in  $\vec{L}^*$ have opposite directions}
%	{
%		%		find two successive edges of the cycle that form a directed path $\dvec{P_2}$ of length 2 with minimal weight\; 
%		\Return the maximum of $h_s(\vec{L}^*)$, and 
%		$\min\{h_s(\dvec{P_2})\mid \dvec{P_2}\mbox{ is a directed path formed by two successive edges of the cycle }\}$;
%	}
%	\lElse{\Return $h_s(\vec{L}^*)$}
%	\caption{BestCostCycle-$s$ $(C)$}
%	\label{algo:cc-s}
%\end{algorithm}
%The inner workings of the algorithm become clearer, though, when an optimal orientation is to be returned, Algorithm \ref{algo:oc-s}.

The algorithm is in the form of a Turing reduction from the cycle $C$ to a certain linear graph
$L$. From an optimal orientation $L^*$ of $L$ it constructs an optimal orientation of $C$.
Its statement uses the following notations.

\noindent {\bf Notation}: To indicate that a given orientation is optimal 
we will add a superscript $^*$.
For example, $\vec{L}^*$ and $\vec{C^*}$ are known optimal orientations of 
the linear graph $L$ and the cycle $C$,
respectively.
\qed
\begin{definition}
	Consider the linear graph $L$ constructed in Algorithm \ref{algo:oc-s}. 
	The orientation $\vec{L}$ of $L$, \emph{induced} by an orientation $\vec{C}$ 
	of $C$ is obtained by equipping edge $\{i,i+1\}$ of $L$ with the direction edge
    $\{i \mod n, (i+1) \mod n\}$ has in $\vec{C}$.
\end{definition}

%The algorithm makes use of several special orientations. 
\begin{definition} Given an optimal orientation $\vec{L}^*$  the algorithm returns one of the 
	following orientations.
	\begin{itemize}
		\item $\vec{C}^{*1way}$ is the one-way orientation of $C$
		whose cost, \emph{OneWayCost-s}, is least.
		\item For a cycle of odd length, $\vec{C}^{2+xx}$ is the orientation of $C$ 
		obtained 
		by first finding a minimum weight path of length 2, and then consecutively alternating the directions of the remaining edges. 
%		Its cost is the maximum of  
%		$$\min\{h_s(\dvec{P_2})\mid \dvec{P_2}\mbox{ is a directed path formed by two successive cycle edges}\}$$ and the maximum of the weights of the cycle edges. 
		See Lemma \ref{l.odd}.
		\item $\vec{C}^{i}$ is the orientation of $C$ obtained by copying 
		from $\vec{L}^*$ the directions
		of the edges $\{k,k+1\},\ i\leq k <i+n$. See Lemma \ref{l.subo}.
		\item 	$\vec{C}^{flip_i}$ is the orientation of $C$ obtained by copying 
		from $\vec{L}^*$ the directions
		of the edges $\{k,k+1\},\ i+1\leq k <i+n$ and the flipped  
		direction of edge $\{i, i+1\}$.
		See Lemma \ref{l.last}.
	\end{itemize}
\end{definition}

\begin{algorithm}
	%		let \textit{OneWayCost-s} be the least cost of a one-way orientation of the cycle, and let $\dvec{C}$ be that orientation\;
	create a linear graph $L$ of 
	length $3n$ by unrolling the cycle three times starting from vertex $0$,
	numbering its vertices \textit{0} to \textit{3n}, and
	equipping edge $\{i,i+1\}$ of $L$ with the same weights as edge 
	$\{i ,i+1\}$ of $C$\;
	let $\vec{L}^*$ be the oriented graph returned by BestOrientLinear-$s(L)$, and $h_s(\vec{L}^*)$ its cost\;
	\lIf{$h_s(\vec{L}^*) \geq$ \textit{OneWayCost-s}}
	{set $\vec{C}$ to $\vec{C}^{*1way}$}
	\lIf{$n$ is odd and every two consecutive edges in  $\vec{L}^*$ have opposite directions}
	{
		%    	find two successive edges of the cycle that form a directed path $\dvec{P_2}$ of length 2 with minimal weight\; 
		%    	construct $\vec{C}$ beginning with $\dvec{P_2}$ and directing each successive edge of the cycle oppositely to its predecessor;
	{set $\vec{C}$ to $\vec{C}^{2+xx}$}
	}
	\lIf{there are two edges $\{i,i+1\}$ and $\{i+n-1,i+n\}$ that
		have opposite directions in $\vec{L}^*$}
	{set $\vec{C}$ to $\vec{C}^{i}$
		%      	\For {$j=i$ to $i+n-1$}
		%	     {copy the direction of edge $\{j,j+1\}$ from $\vec{L}^*$ to $\vec{C}$\;
	}
	\lElse
	%     { \If{there are edges $(i,i+1),(i+1,i+2),(i+3,i+2)$ with $i\leq n$ in $\vec{L}^*$}
	%     	{initialize $\vec{C}$  with edge $(i+1,i)$\;
	%     	\For {$j=i+1$ to $i+n-1$}
	%     	{copy the direction of edge $\{j,j+1\}$ from $\vec{L}^*$ to $\vec{C}$\; }}
	%     	\Else{find edges $(i,i-1),(i-1,i-2),(i-3,i-2)$ with $i\geq n$ in $\vec{L}^*$\;
	%     	          initialize $\vec{C}$  with edge $(i-1,i)$\;
	%               	\For {$j=i-1$ to $i-n+1$}
	%               {copy the direction of edge $\{j,j-1\}$ from $\vec{L}^*$ to $\vec{C}$\; }}
	%       }
	{set $\vec{C}$ to $\vec{C}^{flip_i}$}
	\Return $\vec{C}$;
	\caption{BestOrientCycle-$s$ $(C)$}
	\label{algo:oc-s}
\end{algorithm}

\newpage For ease of reading we break up the proof of correctness of Algorithm \ref{algo:oc-s} into a series of Lemmas. Before doing so we point out that the conditions laid out in 
statements 3 through 6 are exhaustive. In other words, the preconditions of statement 6 are 
that the following hold for $\vec{L}^*$:
\begin{verse}
\texttt{
$h_s(\vec{L}^*) < \textit{OneWayCost-s}$, and \\
every pair of edges $\{j,j+1\}$ and $\{j+n-1,j+n\}$ have the same direction, and \\
there is a directed path of length 2 followed by an edge whose direction is flipped.}
\end{verse}
The first precondition implies that not all edges have the same direction. The last 
precondition states, therefore, that at this point $\vec{L}^*$ is known not to be a completely alternating
orientation: if $n$ is odd because of the falsity of the condition of statement 4, and 
if $n$ is even because of the falsity of the condition of statement 5.

\begin{lemma}\label{l.cd}
	If $\vec{C}$ includes a change in direction then the orientation it induces,  $\vec{L}$, satisfies $h_s(\vec{C})= h_s(\vec{L})$.
\end{lemma}
\noindent {\bf Proof}:
Any orientation of the cycle consists of direction reversals at an even number of vertices 
(including no reversals at all). In particular, if $\vec{C}$  includes a change in direction 
then its longest directed path is
shorter than $n$. Each directed path of the induced orientation $\vec{L}$ is therefore a subpath 
of some path in $\vec{C}$ so that $h_s(\vec{L}) \leq h_s(\vec{C})$, 
by the monotonicity of $h_s$. 
Since $L$ is longer than $2n$, any path of $\vec{C}$ appears in its entirety in $\vec{L}$,
whence also $h_s(\vec{C})\leq h_s(\vec{L})$. 
\qed

Lemma \ref{l.cd} implies the correctness of statement 4 of the algorithm:
\begin{lemma}
	If $h_s(\vec{L}^*) \geq$ \textit{OneWayCost-s} then \textit{OneWayCost-s} is the cost 
	of an optimal orientation of $C$. 
\end{lemma}
\noindent {\bf Proof}: Suppose to the contrary that there is an $\vec{C^*}$ such that 
$h_s(\vec{C^*})<\textit{OneWayCost-s}$. Then there is a direction change in 
$\vec{C^*}$, and the induced orientation $\vec{L}$ satisfies, according to Lemma \ref{l.cd},
$h_s(\vec{L}^*) \leq h_s(\vec{L}) =h_s(\vec{C^*}) <\textit{OneWayCost-s}$,
a contradiction.
\qed

\begin{remark} The orientation $\vec{L}$ induced 
	by $\vec{C^*}$ may not be optimal, even if $\vec{C^*}$ includes a change in direction,
	but this happens only when $n$ is odd and the optimal orientation $\vec{L}^*$ reverses 
	direction at each vertex, see equation (\ref{e.odd}). 
\end{remark}
Here is an example: $n=3$ and $w(i,i+1)=w(i+1,i)=2$ for $i=0,1$,
and  $w(2,0)=w(0,2)=3$. Then  $h_s(\vec{C^*})=4> h_s(\vec{L}^*)=3$.
\qed

\begin{lemma}\label{l.odd}
	Suppose $h_s(\vec{L}^*) < \textit{OneWayCost-s}$. Suppose further that  $n>1$ is odd and that every two consecutive edges in  $\vec{L}^*$ have opposite directions.
	Construct an orientation of the cycle, $\vec{C}^{2+xx}$, as follows:
	\begin{enumerate}
		\item among all possible directed paths of length 2 find one, $\dvec{L_2}$,
		whose weight is minimal; 
		\item 	starting with this path direct each successive edge of the cycle in the direction opposite to that of its predecessor.
	\end{enumerate}
This orientation of the cycle is optimal, and 
\begin{equation}\label{e.odd}
h_s(\vec{C}) = max \{h_s(\vec{L}^*), h_s(\dvec{L_2})\}.
\end{equation}
\end{lemma}
\noindent {\bf Proof}:

Observe first of all that every edge appears in  $\vec{L}^*$ at least once in each of its directions
because the cycle is odd and $L$ contains 3 full cycles. For example, if $(1,2)$ appears in $\vec{L}^*$, then so does
$(n+2,n+1)$. Therefore
$$h_s(\vec{L}^*)= \max_{0\leq i \leq n-1}\{w(i,i+1),w(i+1,i)\}.$$ 
It follows that $h_s(\vec{C})\leq \max \{h_s(\vec{L}^*), h_s(\dvec{L_2})\}$.
Let $\vec{C^*}$ be an optimal orientation of the cycle. Since the cycle is odd $\vec{C^*}$
contains at least two successive edges that form a directed path, so that 
$h_s(\vec{C^*})\geq h_s(\dvec{L_2})$. 

We prove next that there is a change of direction in $\vec{C^*}$. 
Suppose to the contrary that $h_s(\vec{C^*})=\textit{OneWayCost-s}$, and  $\vec{C^*}$
is a one-way orientation, say $(0,1),\ (1,2),\ldots , (n-1,n),\ (n,0)$.
Since $0\leq h_s(\vec{L}^*) < \textit{OneWayCost-s}$ there is an edge with positive weight on $\vec{C}^{*}$, say the edge $(0,1)$. 

Consider first the case that all edges of $\vec{C}^{*}$ have positive weight.
Replacing edge $(0,1)$ with $(1,0)$ we get an orientation 
$\vec{C'}$ with weight $h_s(\vec{C'})=\max \{h_s(\vec{C}^{*})-w(0,1), w(1,0)\}$.
Now 
$$w(1,0)\leq \max_{0\leq i \leq n-1}\{w(i,i+1),w(i+1,i)\}= h_s(\vec{L}^*) <\textit{OneWayCost-s},$$
resulting in the contradiction $\textit{OneWayCost-s}=h_s(\vec{C}^{*})\leq h_s(\vec{C'})<\textit{OneWayCost-s}$.

The same proof idea can also be applied in the general case when some edges are non-positive. For simplicity assume that the edge $(n,0)$ is non-positive, and that 
$i_0=0 <i_1 < \cdots <i_{2k-1}<i_{2k}=n$ are such that all edges $(i,i+1)$ are positive for 
$i_{2\ell}\leq i \leq i_{2\ell+1}$, and non-positive for 
$i_{2\ell+1}\leq i \leq i_{2\ell+2}$, $0\leq \ell \leq k-1$. For each $\ell$ flip an edge
$(j_{\ell},j_{\ell}+1)$ with $i_{2\ell}\leq j_{\ell} \leq i_{2\ell+1}$, and call the resulting orientation $C'$.
Now $h_s(\vec{C'})\leq \max \{h_s(\vec{C}^{*})-\min_{j_{\ell}} w(j_{\ell},j_{\ell}+1), 
\max_{j_{\ell}} w(j_{\ell}+1,j_{\ell})\}$. As before we arrive at the contradiction 
$\textit{OneWayCost-s}=h_s(\vec{C}^{*})\leq h_s(\vec{C'})<\textit{OneWayCost-s}$.

 Lemma \ref{l.cd} ensures, therefore, that the orientation $\vec{L}'$ of $L$ 
induced by $\vec{C^*}$ satisfies $h_s(\vec{C^*})=h_s(\vec{L}')\geq h_s(\vec{L}^*)$.
Hence $h_s(\vec{C}) \leq max \{h_s(\vec{L}^*), h_s(\dvec{L_2})\} \leq h_s(\vec{C^*})$.
\qed

\begin{lemma}\label{l.subo}
	Suppose $h_s(\vec{L}^*) < \textit{OneWayCost-s}$.
	If there is an $i,\ 0\leq i \leq n$, such that the edges $\{i,i+1\}$ and $\{i+n-1,i+n\}$ 
	have opposite directions in $\vec{L}^*$,
	then its subgraph $\vec{L}_{i,i+n}$ induced by the vertices $i,\ldots,i+n$ induces
	an orientation $\vec{C}^{i}$ of the cycle that is optimal, and $h_s(\vec{C}^{i})= h_s(\vec{L}^*)$.
\end{lemma}
\noindent {\bf Proof}:
There is a change of direction at vertex $i$ in $\vec{C}^{i}$,
because the edges $\{i,i+1\}$ and $\{i+n-1,i+n\}$ have opposite directions.
Hence $h_s(\vec{C}^{i}) =h_s(\vec{L}_{i,i+n})\leq h_s(\vec{L}^*)$.

There is also a change of direction in any optimal $\vec{C^*}$, because 
$h_s(\vec{C^*}) \leq h_s(\vec{C}^{i, i+n})  \leq h_s(\vec{L}^*)<\textit{OneWayCost-s}$. 
By Lemma \ref{l.cd},  the orientation $\vec{L'}$ induced by $\vec{C}^*$
satisfies $h_s(\vec{L'})= h_s(\vec{C^*})$. Hence
$$h_s(\vec{L}^*)\leq h_s(\vec{L'})= h_s(\vec{C^*})\leq h_s(\vec{L}^*) .$$

\qed

\begin{lemma}\label{l.last}
	Suppose $\vec{L}^*$ satisfies the following preconditions 
	\begin{enumerate}
		\item $h_s(\vec{L}^*) < \textit{OneWayCost-s}$;
		\item \label{i.o1}every pair of edges $\{j,j+1\}$ and $\{j+n-1,j+n\}$, $0\leq j \leq 2n$,
		 have the same direction in $\vec{L}^*$;
		\item there are two consecutive edges in  $\vec{L}^*$ that point in the same direction.
	\end{enumerate} 
Construct an orientation of the cycle, $\vec{C}^{flip_i}$ as follows:
\begin{enumerate}
	\item find three consecutive edges in $\vec{L}^*$ such that the first two point in the same direction and the third points in the opposite direction;
	\item add the first edge to $\vec{C}^{flip_i}$ but reverse its direction;
	\item starting with the second edge copy the directions of the following $n-1$ edges 
	from $\vec{L^*}$.
\end{enumerate}
This orientation is optimal and $h_s(\vec{L}^*) = h_s(\vec{C}^{flip_i})$.
	\end{lemma}
\noindent {\bf Proof}: 
It is possible to find the three consecutive edges of step 1 of the construction, because 
all paths in $\vec{L}^*$ are shorter than $n$ since $h_s(\vec{L}^*) < \textit{OneWayCost-s}$.
The first two edges in such a triple can be directed to the left or to the right.
For simplicity and w.l.o.g we assume that they are $(0,1), (1,2),(3,2)$. Precondition
	 \ref{i.o1} ensures that $\vec{L}^*$ also contains the edges $(n-1,n),(n,n+1),(n+2,n+1)$ and $(2n-2,2n-1),(2n-1,2n), (2n+1,2n) $, from which it follows that $\vec{C}^{flip_i}$ has edges 
	 $(1,0), (1,2),(3,2),(n-1,0)$, among others.
	
	We show next that any directed path in $\vec{C}^{flip_i}$ appears in $\vec{L}^*$. Consider first a directed path $\vec{P}$ that contains the edge $(1,0)$. 
	Since this edge is surrounded by the edges $(1,2), (n-1,0)$ it is the only one in $\vec{P}$. Moreover, edge $(1,0)$ is identical to edge $(2n+1,2n)$, an edge that belongs to 
	$\vec{L}^*$, proving that  $\vec{P}$ is a path in $\vec{L}^*$.
	Any other path in $\vec{C}^{flip_i}$ does not contain the edge $(1,0)$ and is therefore by its construction a path in $\vec{L}^*$. 
	Consequently, $h_s(\vec{C}^{flip_i})\leq h_s(\vec{L}^*)<\textit{OneWayCost-s}$.
	
	Let $\vec{C}^*$ be an optimal orientation of the cycle. It contains a direction reversal
	because $h_s(\vec{C}^*)\leq h_s(\vec{C}^{flip_i})<\textit{OneWayCost-s}$.
	Hence the orientation of $L$ it induces, $\vec{L}'$, satisfies
	$h_s(\vec{C}^*)=h_s(\vec{L}')\geq h_s(\vec{L}^*)\geq h_s(\vec{C}^{flip_i})$.
\qed

We summarize the previous analysis as a Theorem.
\begin{theorem}
	BestOrientCycle-$s$  returns  in linear time an orientation for a cycle graph that is optimal under the cost function $h_s$.
\end{theorem}
%\noindent {\bf Proof}: The algorithm reduces the problem of finding an optimal 
%orientation of the cycle to finding one on the linear graph $L$.

Another conclusion from Lemmas \ref{l.odd}, \ref{l.subo} and \ref{l.last} is
\begin{corollary}
	If $h_s(\vec{L}^*) < {OneWayCost_s}$ then $h_s(\vec{L}^*)$ is the cost of an optimal orientation of the cycle, unless $n$ is odd and $\vec{L}^*$ reverses direction at each vertex.
\end{corollary}

\subsection{An algorithm for cost function $H_m$}\label{s.c}

The cycle algorithm for cost function $H_m$ is, disappointingly, the simplest one: 
try all possibilities of breaking up the cycle into a linear graph, applying 
BestCostLinear$_m$ to that graph, and choose the one with least cost. 
To describe th algorithm in more detail 
we denote by $\vec{C}^{*1way}$ the one-way orientation of $C$
whose cost, ${OneWayCost_m}$, is least; and by $L_{i+1,i-1}$ for $0\leq i \leq n$ 
the subgraph of $C$
that is induced by vertices $i+1,i+2,\ldots, i-1$. Recall that sums are 
always to be taken mod $n$.
\begin{algorithm}
set $BestCost$ to $OneWayCost_m$
%, and set $ \vec{C}$ to $\vec{C}^{*1way}$
\;
	\label{ac.i1} \For{$i= 0$ to $n-1$}{	
construct
a linear graph  $L^i$ by adding two vertices $i'$ and $i''$ to $L_{i+1,i-1}$ with edge weights
$w_L(i',i+1)=w(i,i+1)$,\ $w_L(i+1,i')=w_L(i-1,i'')=\infty$ and 
$w_L(i'',i-1)=w(i,i-1)$\;
			\If{BestCostLinear$_m(L^i)< BestCost$}
			{set $BestCost$ to BestCostLinear$_m(L^i)$\;
%		 	 set $ \vec{C}$ to the orientation of $C$  induced by $L$;
			}  		
	}
	\Return $BestCost$; 
	\caption{BestCostCycle$_m(C)$}
	\label{algo:oc-m}
\end{algorithm}

\bigskip
\begin{theorem}
\emph{Algorithm BestCostCycle}$_m(C)$ returns in $O(n^2 \log n)$ time the cost of an optimal 
orientation of $C$.
\end{theorem}
\begin{proof}

Let $C^*$ be an optimal orientation 
of $C$.
If $C^*$ is a clockwise or counterclockwise cycle then the opptimal cost will be found in step 1.
Otherwise, $C^*$ contains at least one vertex $i$ such that both of the edges $\{i-1,i\}$ and $\{i,i+1\}$
are oriented away from \textit{i} in $C^*$.  Consider the linear graph $L^i$ constructed in
the $i$-the execution of the loop \ref{ac.i1}. Denote by $\vec{L^i}$ the orientation of $L^i$ 
induced by $C^*$  (by breaking the cycle at vertex $i$), and note that 
$h_m(\vec{L^i})=h_m(\vec{C^*})$. Let $\vec{L^{i*}}$ be an optimal orientation of $L^i$.
$\vec{L^{i*}}$ induces an orientation $\vec{C}$ of $C$ (by identifying $i'$ with $i''$)
and $h_m(\vec{C})=h_m(\vec{L^{i*}})$.  
Then $h_m(\vec{C})=h_m(\vec{L^{i*}})\leq h_m(\vec{L^i})=h_m(\vec{C^*})$.
Since $h_m(\vec{C})\geq h_m(\vec{C^*})$ it follows that $h_m(\vec{L^{i*}})= h_m(\vec{C^*})$.

Because one call to $BestCostPath(P)$ takes $O(n \log n)$ time the algorithm runs in $O(n^2 \log n)$ time.
\end{proof}
% bibliography, glossary and index would go here.

\bibliography{orientation.bib}{}
\bibliographystyle{plain}

%\begin{thebibliography}{99}
%	\bibitem{chen2004} Chen, Kuan-Yu and Chao, Kun-Mao,
%	On the range maximum-sum segment query problem,
%	{\it Discrete Applied Mathematics}, 155, pp. 2043-- 2052, 2007.
%	%@inproceedings{chen2004range,
%	%	title={},
%	%	author={Chen, Kuan-Yu and Chao, Kun-Mao},
%	%	booktitle={International Symposium on Algorithms and Computation},
%	%	pages={294--305},
%	%	year={2004},
%	%	organization={Springer}
%	%}
%\end{thebibliography}

\end{document}