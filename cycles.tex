\section{Algorithms for cycles}\label{s.c}
Given a cycle graph $C$ on $n$ vertices, number its vertices from \textit{0} to \textit{n-1}, 
clockwise from an arbitrary node. 
Denote the weight of a directed
edge $(i,i+1)$ by $w(i,i+1)$. When a node $j$ with $j\geq n$ is referred to it 
should be understood as referring to node $j \mod n$. 
For example, $w(n-1,n)=w(n-1,0)$.

\subsection{An algorithm for cost function $H_s$}


The algorithm is in the form of a Turing reduction from the cycle $C$ to a certain linear graph
$L$. From an optimal orientation $\tildeb{L^*}$ of $L$ it constructs an optimal orientation of $C$.
Its statement uses the following notations.

\noindent {\bf Notation}: To indicate that a given orientation is optimal 
we will add a superscript $^*$.
For example, $\tildeb{L^*}$ and $\tildeb{C^*}$ are known optimal orientations of 
the linear graph $L$ and the cycle $C$,
respectively.
\qed
\begin{definition}
	Consider the linear graph $L$ constructed in Algorithm \ref{algo:oc-s}. 
	The orientation $\tildeb{L}$ of $L$, \emph{induced} by an orientation $\tildeb{C}$ 
	of $C$ is obtained by equipping edge $\{i,i+1\}$ of $L$ with the direction edge
    $\{i \mod n, (i+1) \mod n\}$ has in $\tildeb{C}$.
\end{definition}

\begin{definition} Given an optimal orientation $\tildeb{L}^*$  the algorithm returns one of the 
	following orientations.
	\begin{itemize}
		\item $\tildeb{C}^{*1way}$ is the one-way orientation of $C$
		whose cost, $OneWayCost_s$, is least.
		\item For a cycle of odd length, $\tildeb{C}^{2+xx}$ is the orientation of $C$ 
		obtained as follows:
		among all possible directed paths of length 2 find one, $\dvec{L_2}$,
		whose weight is minimal; 
	starting with this path direct each successive edge of the cycle in the direction opposite to that of its predecessor.
		See Lemma \ref{l.odd}.
		\item $\tildeb{C}^i$ is the orientation of $C$ obtained by copying 
		from $\tildeb{L}^*$ the directions
		of the edges $\{k,k+1\},\ i\leq k <i+n$. See Lemma \ref{l.subo}.
		\item 	$\tildeb{C}^{rflip_i}$ is an orientation of $C$ that is created when $0\leq i \leq 2n$ and
		$\tildeb{L}^*$ contains the directed edges 
		$(i,i+1),(i+1,i+2),(i+3,i+2)$. Its first directed edge
		is $(i+1,i)$, and the directions of the following edges $\{k,k+1\},\ i+1\leq k <i+n$
		are copied from $\tildeb{L}^*$.
		See Lemma \ref{l.last}.
		\item 	$\tildeb{C}^{lflip_i}$ is an orientation of $C$ that is created when $n\leq i \leq 3n$ and
$\tildeb{L}^*$ contains the directed edges 
$(i,i-1),(i-1,i-2),(i-3,i-2)$. Its first directed edge
is $(i-1,i)$, and the directions of the following edges $\{k,k-1\},\ i-n<k \leq i-1$,
are copied from $\tildeb{L}^*$.
See Lemma \ref{l.last}.
	\end{itemize}
\end{definition}

\begin{algorithm}
	 	\KwIn{a cycle graph $C$ on $n$ vertices}
	\KwOut{an optimal orientation of $C$ under $H_s$}
	create a linear graph $L$ of 
	length $3n$ by unrolling the cycle three times starting from vertex $0$,
	numbering its vertices \textit{0} to \textit{3n}, and
	equipping edge $\{i,i+1\}$ of $L$ with the same weights as edge 
	$\{i ,i+1\}$ of $C$\;
	let $\tildeb{L}^*$ be the oriented graph returned by BestOrientLinear$_s(L)$, and $h_s(\tildeb{L}^*)$ its cost\;
	\lIf{$h_s(\tildeb{L}^*) \geq OneWayCost_s$}
	{set $\tildeb{C}$ to $\tildeb{C}^{*1way}$}
	\lElseIf{$n$ is odd and every two consecutive edges in  $\tildeb{L}^*$ have opposite directions}
	{
	{set $\tildeb{C}$ to $\tildeb{C}^{2+xx}$}
	}
	\lElseIf{there are two edges $\{i,i+1\}$ and $\{i+n-1,i+n\}$ that
		have opposite directions in $\tildeb{L}^*$}
	{set $\tildeb{C}$ to $\tildeb{C}^{i}$
	}\label{s.flipi}
	\lElseIf{there is an $i$, $0\leq i\leq 2n$ such that 
		the directed edges $(i,i+1),(i+1,i+2),(i+3,1+2)$ appear in $\tildeb{L}^*$ }{set $\tildeb{C}$ to $\tildeb{C}^{rflip_i}$}\label{s.pilfi}
	\lElse
	{set $\tildeb{C}$ to $\tildeb{C}^{lflip_i}$}
	\Return $\tildeb{C}$;
	\caption{BestOrientCycle$_s(C)$}
	\label{algo:oc-s}
\end{algorithm}

For convenience we spell out what we know about $\tildeb{L}^*$ when the conditions of statements
\ref{s.flipi} and \ref{s.pilfi} are checked.
\begin{lemma}\label{l.flipc}
	If the conditions of statements 3, 4, and 5 do not hold  then $\tildeb{L}^*$ has the following properties:
	\begin{itemize}
		\item for $0\leq i \leq 2n$, the edges $\{i,i+1\}$ and $\{i+n-1,i+n\}$ have the same direction;
		\item either there is an $i$, $0\leq i\leq 2n$, such that 
		the edges $(i,i+1),(i+1,i+2),(i+3,i+2)$ belong to $\tildeb{L}^*$, or
		there is an $i$, $n\leq i\leq 3n$, such that the edges 
		$(i,i-1),(i-1,i-2),(i-3,i-2)$ belong to $\tildeb{L}^*$. 
	\end{itemize}
\end{lemma}
\begin{proof}
	The first item is self evident from the failure of statement 5. 
	
	To prove the second item note 
	that $h_s(\tildeb{L}^*)<\mbox{OneWayCost}_s$ implies
	that each directed path in $\tildeb{L}^*$ has length less than $n$. Moreover, from
	statements 4 and 5 it follows that not all successive edges of $\tildeb{L}^*$ alternate in direction,
	i.e. there are at least two successive edges with the same direction. Consequently, 
	$\tildeb{L}^*$ contains either a triplet of edges $(i,i+1),(i+1,i+2),(i+3,1+2)$ or a triplet of edges 
	$(i,i-1),(i-1,i-2),(i-3,1-2)$. 
	
	We show that $0\leq i\leq 2n$ in the former case; the proof that $n\leq i\leq 3n$ in the latter case
	is entirely similar. Suppose there is a triplet $(i,i+1),(i+1,i+2),(i+3,i+2)$ with $2n< i \leq 3n-3$. 
	It follows then from the first item that $\tildeb{L}^*$ also contains the triplet $(i-n+1,i-n+2),(i-n+2,i-n+3),(i-n+4,i-n+3)$, and $n+1 < i-n+1 \leq 2n-2$.
	
	
\end{proof}

For ease of reading we break up the proof of correctness of Algorithm \ref{algo:oc-s} into a series of Lemmas. 

\begin{lemma}\label{l.cd}
	If $\tildeb{C}$ includes a change in direction then the orientation it induces,  $\tildeb{L}$, satisfies $h_s(\tildeb{C})= h_s(\tildeb{L})$.
\end{lemma}
\noindent {\bf Proof}:
Any orientation of the cycle consists of direction reversals at an even number of vertices 
(including no reversals at all). In particular, if $\tildeb{C}$  includes a change in direction 
then its longest directed path is
shorter than $n$. Each directed path of the induced orientation $\tildeb{L}$ is therefore a subpath 
of some path in $\tildeb{C}$ so that $h_s(\tildeb{L}) \leq h_s(\tildeb{C})$, 
by the monotonicity of $h_s$. 
Since $L$ is longer than $2n$, any path of $\tildeb{C}$ appears in its entirety in $\tildeb{L}$,
whence also $h_s(\tildeb{C})\leq h_s(\tildeb{L})$. 
\qed

Lemma \ref{l.cd} implies the correctness of statement 4 of the algorithm:
\begin{lemma}
	If $h_s(\tildeb{L}^*) \geq OneWayCost_s$ then $OneWayCost_s$ is the cost 
	of an optimal orientation of $C$. 
\end{lemma}
\noindent {\bf Proof}: Suppose to the contrary that there is a $\tilde{C}^*$ such that 
$h_s(\tildeb{C}^*)< OneWayCost_s$. Then there is a direction change in 
$\tildeb{C^*}$, and the induced orientation $\tildeb{L}$ satisfies, according to Lemma \ref{l.cd},
$h_s(\tilde{L}^*) \leq h_s(\tildeb{L}) =h_s(\tildeb{C}^*) < OneWayCost_s$,
a contradiction.
\qed

\begin{remark} The orientation $\tildeb{L}$ induced 
	by $\tildeb{C}^*$ may not be optimal, even if $\tildeb{C}^*$ includes a change in direction,
	but this happens only when $n$ is odd and the optimal orientation $\tilde{L}^*$ reverses 
	direction at each vertex, see equation (\ref{e.odd}). 
\end{remark}
Here is an example: $n=3$ and $w(i,i+1)=w(i+1,i)=2$ for $i=0,1$,
and  $w(2,0)=w(0,2)=3$. Then  $h_s(\tildeb{C^*})=4> h_s(\tildeb{L}^*)=3$.


\begin{lemma}\label{l.odd}
	Suppose $h_s(\tildeb{L}^*) < OneWayCost_s$. Suppose further that  $n>1$ is odd and that every two consecutive edges in  $\tildeb{L}^*$ have opposite directions.
Then the cycle orientation $\tildeb{C}^{2+xx}$ is optimal, and 
\begin{equation}\label{e.odd}
h_s(\tildeb{C}) = max \{h_s(\tildeb{L}^*), h_s(\dvec{L_2})\}.
\end{equation}
\end{lemma}
\noindent {\bf Proof}:

Observe first of all that every edge appears in  $\tildeb{L}^*$ at least once in each of its directions
because the cycle is odd and $L$ contains 3 full cycles. For example, if $(1,2)$ appears in $\tildeb{L}^*$, then so does
$(n+2,n+1)$. Therefore
$$h_s(\tildeb{L}^*)= \max_{0\leq i \leq n-1}\{w(i,i+1),w(i+1,i)\}.$$ 
It follows that $h_s(\tildeb{C})\leq \max \{h_s(\tildeb{L}^*), h_s(\dvec{L_2})\}$.
Let $\tildeb{C}^*$ be an optimal orientation of the cycle. Since the cycle is odd $\tildeb{C}^*$
contains at least two successive edges that form a directed path, so that 
$h_s(\tildeb{C^*})\geq h_s(\dvec{L_2})$. 

We prove next that there is a change of direction in $\tildeb{C^*}$. 
Suppose to the contrary that $h_s(\tildeb{C}^*)=OneWayCost_s$, and  $\tildeb{C}^*$
is a one-way orientation, say $(0,1),\ (1,2),\ldots , (n-1,n),\ (n,0)$.
Since $0\leq h_s(\tildeb{L}^*) <  OneWayCost_s$ there is an edge with positive weight on $\tildeb{C}^{*}$, say the edge $(0,1)$. 

Consider first the case that all edges of $\tildeb{C}^{*}$ have positive weight.
Replacing edge $(0,1)$ with $(1,0)$ we get an orientation 
$\tildeb{C'}$ with weight $h_s(\tildeb{C'})=\max \{h_s(\tildeb{C}^{*})-w(0,1), w(1,0)\}$.
Now 
$$w(1,0)\leq \max_{0\leq i \leq n-1}\{w(i,i+1),w(i+1,i)\}= h_s(\tildeb{L}^*) <OneWayCost_s,$$
resulting in the contradiction $OneWayCost_s=h_s(\tildeb{C}^{*})\leq h_s(\tildeb{C'})<OneWayCost_s$.

The same proof idea can also be applied in the general case when some edges are non-positive. For simplicity assume that the edge $(n-1,0)$ is non-positive, and that 
$i_0=0 <i_1 < \cdots <i_{2k-1}<i_{2k}=n$ are such that all edges $(i,i+1)$ are positive for 
$i_{2\ell}\leq i \leq i_{2\ell+1}$, and non-positive for 
$i_{2\ell+1}\leq i \leq i_{2\ell+2}$, $0\leq \ell \leq k-1$. For each $\ell$ flip an edge
$(j_{\ell},j_{\ell}+1)$ with $i_{2\ell}\leq j_{\ell} \leq i_{2\ell+1}$, and call the resulting orientation $C'$.
Now $h_s(\tildeb{C'})\leq \max \{h_s(\tildeb{C}^{*})-\min_{j_{\ell}} w(j_{\ell},j_{\ell}+1), 
\max_{j_{\ell}} w(j_{\ell}+1,j_{\ell})\}$. As before we arrive at the contradiction 
$OneWayCost_s=h_s(\tildeb{C}^{*})\leq h_s(\tildeb{C'})< OneWayCost_s$.

 Lemma \ref{l.cd} ensures, therefore, that the orientation $\tildeb{L}'$ of $L$ 
induced by $\tildeb{C^*}$ satisfies $h_s(\tildeb{C^*})=h_s(\tildeb{L}')\geq h_s(\tildeb{L}^*)$.
Hence $h_s(\tildeb{C}) \leq max \{h_s(\tildeb{L}^*), h_s(\dvec{L_2})\} \leq h_s(\tildeb{C^*})$.
\qed

\begin{lemma}\label{l.subo}
	Suppose $h_s(\tildeb{L}^*) < OneWayCost_s$.
	If there is an $i,\ 0\leq i \leq n$, such that the edges $\{i,i+1\}$ and $\{i+n-1,i+n\}$ 
	have opposite directions in $\tildeb{L}^*$,
	then its subgraph $\tildeb{L}_{i,i+n}$ induced by the vertices $i,\ldots,i+n$ induces
	an orientation $\tildeb{C}^{i}$ of the cycle that is optimal, and $h_s(\tildeb{C}^{i})= h_s(\tildeb{L}^*)$.
\end{lemma}
\begin{proof}
	
There is a change of direction at vertex $i$ in $\tildeb{C}^{i}$,
because the edges $\{i,i+1\}$ and $\{i+n-1,i+n\}$ have opposite directions.
Hence $h_s(\tildeb{C}^{i}) =h_s(\tildeb{L}_{i,i+n})\leq h_s(\tildeb{L}^*)$.

There is also a change of direction in any optimal $\tildeb{C}^*$, because 
$h_s(\tildeb{C^*}) \leq h_s(\tildeb{C}^{i, i+n})  \leq h_s(\tildeb{L}^*)<OneWayCost_s$. 
By Lemma \ref{l.cd},  the orientation $\tildeb{L'}$ induced by $\tildeb{C}^*$
satisfies $h_s(\tildeb{L'})= h_s(\tildeb{C}^*)$. Hence
$$h_s(\tildeb{L}^*)\leq h_s(\tildeb{L'})= h_s(\tildeb{C}^*)\leq h_s(\tildeb{L}^*) .$$
\end{proof}

Finally we consider the orientations constructed when the conditions of statements 3, 4, and 5 
do not hold. Recall that in this case Lemma \ref{l.flipc} guarantees the existence of a triplet
of 3 successive edges the first pointing to the second, the second pointing to the third and the last pointing back to the second.
\begin{lemma}\label{l.last}
	Suppose that $h_s({L}^*) < OneWayCost_s$ and that 
	all edges $\{i,i+1\}$ and $\{i+n-1,i+n\}$, $0\leq i \leq 2n$, have the same direction in $\tildeb{L}^*$.
	If $\tildeb{L}^*$  contains a triplet of edges  $(i,i+1),(i+1,i+2),(i+3,i+2)$, $0\leq i\leq 2n$, then 
	$\tildeb{C}^{rflip_i}$ is an optimal orientation and $h_s(\tildeb{L}^*) = h_s(\tildeb{C}^{rflip_i})$;
	and if it contains a triplet of edges
	$(i,i-1),(i-1,i-2),(i-3,i-2)$, $n\leq i\leq 3n$,  then $\tildeb{C}^{lflip_i}$ 
	is an optimal orientation and $h_s(\tildeb{L}^*) = h_s(\tildeb{C}^{lflip_i})$.
	\end{lemma}
\begin{proof}
We prove the optimality of 	$\tildeb{C}^{rflip_i}$.
For simplicity and w.l.o.g we assume that $i=0$, i.e. the triplet is $(0,1), (1,2),(3,2)$. 
Since edges $\{i,i+1\}$ and $\{i+n-1,i+n\}$ have the same direction
%Precondition \ref{i.o1} ensures that $\tildeb{L}^*$ also contains 
the edges $(n-1,n),(n,n+1),(n+2,n+1)$ and $(2n-2,2n-1),(2n-1,2n), (2n+1,2n) $ 
also belong to $\tildeb{L}^*$,
from which it follows that $\tildeb{C}^{rflip_i}$ has edges 
	 $(1,0), (1,2),(3,2),(n-1,0)$, among others.
	
	We show next that any directed path in $\tildeb{C}^{flip_i}$ appears in $\tildeb{L}^*$. Consider first a directed path $\vec{P}$ that contains the edge $(1,0)$. 
	Since this edge is surrounded by the edges $(1,2), (n-1,0)$ it is the only one in $\vec{P}$. Moreover, edge $(1,0)$ is identical to edge $(2n+1,2n)$, which belongs to 
	$\tildeb{L}^*$, proving that  $\vec{P}$ is a path in $\tildeb{L}^*$.
	Any other path in $\tildeb{C}^{rflip_i}$ does not contain the edge $(1,0)$ and is therefore 
	also a path in $\tildeb{L}^*$. 
	Consequently, $h_s(\tildeb{C}^{rflip_i})\leq h_s(\tildeb{L}^*)<OneWayCost_s$.
	
	Let $\tildeb{C}^*$ be an optimal orientation of the cycle. It contains a direction reversal
	because $h_s(\tildeb{C}^*)\leq h_s(\tildeb{C}^{rflip_i})<OneWayCost_s$.
	Hence the orientation of $L$ it induces, $\tildeb{L}'$, satisfies
	$h_s(\tildeb{C}^*)=h_s(\tildeb{L}')\geq h_s(\tildeb{L}^*)\geq h_s(\tildeb{C}^{rflip_i})$.
\end{proof}

We summarize the previous analysis as a Theorem.
\begin{theorem}\label{t.cycle-s}
	\emph{BestOrientCycle}$_s$  returns  in linear time an orientation for a cycle graph that is optimal under the cost function $H_s$.
\end{theorem}

Another conclusion from Lemmas \ref{l.odd}, \ref{l.subo} and \ref{l.last} is
\begin{corollary}
	If $h_s(\tildeb{L}^*) < OneWayCost_s$ then $h_s(\tildeb{L}^*)$ is the cost of an optimal orientation of the cycle, unless $n$ is odd and $\tildeb{L}^*$ reverses direction at each vertex.
\end{corollary}

\subsection{An algorithm for cost function $H_m$}\label{s.cm}

To our disappointment we have been unable to devise a cycle algorithm for cost function $H_m$ that 
is more efficient than the simplest one: 
out of all possibilities of breaking up the cycle into a linear graph, and applying 
BestCostLinear$_m$ to that graph, choose the one with least cost. 
To describe th algorithm in more detail 
we denote by $\tildeb{C}^{*1way}$ the one-way orientation of $C$
whose cost, ${OneWayCost_m}$, is least; and by $L_{i+1,i-1}$ for $0\leq i \leq n$ 
the subgraph of $C$
that is induced by vertices $i+1,i+2,\ldots, i-1$. Recall that sums are 
always to be taken mod $n$.
\begin{algorithm}
		 	\KwIn{a cycle graph $C$ on $n$ vertices}
	\KwOut{an optimal orientation of $C$ under $H_m$}
set $BestCost$ to $OneWayCost_m$
, and set $ \tildeb{C}$ to $\tildeb{C}^{*1way}$
\;
	\label{ac.i1} \For{$i= 0$ to $n-1$}{	
construct
a linear graph  $L^i$ by adding two vertices $i'$ and $i''$ to $L_{i+1,i-1}$ with edge weights
$w_L(i',i+1)=w(i,i+1)$,\ $w_L(i+1,i')=w_L(i-1,i'')=\infty$ and 
$w_L(i'',i-1)=w(i,i-1)$\;
let $\tildeb{L}^i$ be the 
%oriented linear 
graph returned by $BestOrientLinear_m(L^i)$\;
			\If{$h_m(\tildeb{L}^i)< BestCost$}
			{set $\tildeb{C}$ to the orientation of $C$ induced by $\tildeb{L}^i$;
			}  		
	}
	\Return $\tildeb{C}$; 
	\caption{BestOrientCycle$_m(C)$}
	\label{algo:oc-m}
\end{algorithm}

\bigskip
\begin{theorem}\label{t.cycle-m}
\emph{Algorithm BestOrientCycle}$_m(C)$ returns in $O(n^2 \log n)$ time an optimal 
orientation of $C$.
\end{theorem}
\begin{proof}

Let $\tildeb{C}^*$ be an optimal orientation 
of $C$.
If $\tildeb{C}^*$ is a clockwise or counterclockwise cycle then the opptimal cost will be found in step 1.
Otherwise, $\tildeb{C}^*$ contains at least one vertex $i$ such that both of the edges $\{i-1,i\}$ and $\{i,i+1\}$
are oriented away from \textit{i} in $C^*$.  Consider the linear graph $L^i$ constructed in
the $i$-the execution of the loop \ref{ac.i1}. Denote by $\tildeb{L^i}$ the orientation of $L^i$ 
induced by $C^*$  (by breaking the cycle at vertex $i$), and note that 
$h_m(\tildeb{L}^i)=h_m(\tildeb{C^*})$. Let $\tildeb{L}^{i*}$ be an optimal orientation of $L^i$.
$\tildeb{L}^{i*}$ induces an orientation $\tildeb{C}$ of $C$ (by identifying $i'$ with $i''$)
and $h_m(\tildeb{C})=h_m(\tildeb{L^{i*}})$.  
Then $h_m(\tildeb{C})=h_m(\tildeb{L^{i*}})\leq h_m(\tildeb{L^i})=h_m(\tildeb{C^*})$.
Since $h_m(\tildeb{C})\geq h_m(\tildeb{C^*})$ it follows that $h_m(\tildeb{L^{i*}})= h_m(\tildeb{C^*})$.

Because one call to $BestCostPath(P)$ takes $O(n \log n)$ time the algorithm runs in $O(n^2 \log n)$ time.
\end{proof}