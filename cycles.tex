\section{An algorithm for cycles}\label{s.c}
Given a cycle graph $C_n$ on $n$ vertices, number its vertices from \textit{0} to \textit{n-1}, 
clockwise from an arbitrary node. 
Denote the weight of a directed
edge $(i,i+1)$ by $w(i,i+1)$. When a node $j$ with $j\geq n$ is referred to it 
should be understood as referring to node $j \mod n$. 
For example, $w(n-1,n)=w(n-1,0)$.

\begin{theorem}
	There is a linear time algorithm for finding an orientation for a cycle graph that is optimal	under the cost function $W_s$.
\end{theorem}
\noindent {\bf Proof}:

\qed
%The following algorithm uses the BestCostPath algorithm as a building block. 
%
%\noindent \textbf{Algorithm BestCostCycle ($C_n$)}:
%
%\begin{enumerate}
%	\item $BestCost \leftarrow \min\{\sum_{i=0}^{n-1}w(i,i+1), \sum_{i=0}^{n-1}w(i+1,i)\}$;
%	\item \label{ac.i1} for $i= 0$ to $n-1$ do:
%	\begin{enumerate}
%		\item break the cycle at vertex $i$ by making two copies of vertex 
%		$i$, $i'$ and $i''$, and setting $P$ as the path $i', i+1,\ldots ,0,\ldots i-1, i''$
%		with weights the same as in $C_n$ except for the edges involving $i', i''$;  
%		\item \label{i.c1}let $P^{out}$ be the path $P$
%		with $w(i',i+1)=w(i,i+1),\ w(i+1,i')=\infty$ and 
%		$w(i-1,i'')=\infty,\  w(i'',i-1)=w(i,i-1)$;
%		\item \label{i.c2}
%%		$Cost(P_1)\leftarrow BestCostPath(P_1)$; 
%		if $BestCostPath(P^{out})< BestCost$ then update $BestCost$;
%		\item \label{i.c3} let $P^{in}$ be the path $P$
%		with $w(i',i+1)=\infty,\ w(i+1,i')=w(i+1,i)$ and 
%		$w(i-1,i'')=w(i-1,i),\  w(i'',i-1)=\infty$;
%		\item \label{i.c4}
%%		$Cost(P_2)\leftarrow BestCostPath(P_2)$; 
%		if $BestCostPath(P^{in})< BestCost$ then update $BestCost$.
%	\end{enumerate}		
%	\item return $BestCost$; 
%\end{enumerate}	
%\noindent \textbf{End of Algorithm BestCostCycle}
%\bigskip
%
%To prove the correctness of the algorithm let \textit{O} be an optimal orientation 
%of $C_n$.
%If \textit{O} is a clockwise or counterclockwise cycle then it will be found in step 1.
%Otherwise, let \textit{i} be a vertex such that either both of the edges $\{i-1,i\}$ and $\{i,i+1\}$
%are oriented away from \textit{i} in \textit{O}, or both of them are oriented towards \textit{i}.
%In the former case, any optimal orientation of the path $P^{out}$ formed in 
%the $i$-the execution of the loop yields an
%orientation of $C_n$, because both edges involving $i$ are outward oriented
%by the construction of $P^{out}$ in step \ref{i.c1}. This orientation has the same cost 
%as \textit{O}, and it will be found in step \ref{i.c2}.
%Similarly, in the latter case, any optimal orientation of $P^{in}$ yields an optimal
%orientation of $C_n$, and it will be found in step \ref{i.c4}.
%
%Since a call to $BestCostPath(P)$ takes $O(n^2)$ time the algorithm runs in $O(n^3)$ time.
