 \section{Algorithms for stars}\label{s.1}
 A star $S_k$ is a tree with one internal node $c$ and $k$ leaves.

 \subsection{An algorithm for cost function $H_m$}
 \begin{algorithm}
Orient each edge $\{u,c\}$ inwards to \textit{c} if $w(u,c)<w(c,u)$,
and outward from $c$ otherwise\; \label{i0}
 	 
denote by $\vec{S_n}$ the resulting directed graph, by $BestCost$ its cost, 
by $E_{in}=\{(u_1,c),\ldots, (u_{\ell},c)\}$ the list of its inward edges, 
and by $E_{out}=\{(c,v_1),\ldots, (c,v_r)\}$ the list of its outward edges\;
\label{i00} reorder each of $E_{in}$ and $E_{out}$ so that
the weights of its edges are in non-increasing order\;

set $\vec{S}$ to $\vec{S_n}$\;

\For {$k= 1$ to $\ell$}
  {update $\vec{S}$ by flipping the direction of edge $(u_k,c)$, and update $BestCost$ if necessary}
   \label{i1} 
set $\vec{S}$ to $\vec{S_n}$\;
\For {$k= 1$ to $r$}
   {update $\vec{S}$ by flipping the direction of edge $(c,v_k)$, and update $BestCost$ if necessary}
  \label{i2}
 	\Return an orientation whose cost is $BestCost$\;
 	\caption{Algorithm BestOrientStar-m ($S_n$)}
 	\label{algo:oc-s}
 \end{algorithm}

\bigskip

To establish the correctness of the algorithm we first prove its key observation:  
if initially  each edge is oriented
so that it points in its lighter direction then an optimal orientation can be
found by only flipping edges that were initially pointing inwards or only
flipping edges that were initially pointing outwards. 
\begin{lemma}\label{l.best}
	There is no optimal orientation in which both the largest inward edge weight is less than $w(u_1,c)$ 
	and the largest outward edge weight is less than $w(c,v_1)$.
	%, where $(u_1,c)$ is an edge with largest weight in $E_{in}$, and $(c,v_1)$ an edge with largest weight in $E_{out}$.
\end{lemma}

\noindent \textbf{Proof:}
Note that the cost of the initial orientation is $h_m(\vec{S_n})=w(u_1,c)+w(c,v_1)$.
Suppose, by contradiction, that there is an optimal orientation $\vec{S^*}$ 
with largest inward weight $w(x,c)< w(u_1,c)$,
and largest outward weight $w(c,y)<w(c,v_1)$.
Then $h_m(\vec{S^*})=w(x,c)+w(c,y)$.
The initial orientation implies that $w(c,u_1) > w(u_1,c)$, and $w(v_1,c) \geq w(c,v_1)$.
Since $\{u_1,c\}$ is an outward edge in $\vec{S^*}$ and $\{v_1,c\}$ an inward edge,
$h_m(\vec{S^*})\geq  
w(v_1,c)+w(c,u_1) >  w(u_1,c)+w(c,v_1) =h_m(\vec{S_n})$, a contradiction.
\qed

\begin{theorem}
%	[Correctness of the algorithm]
Algorithm BestOrientStar-m finds an optimal orientation in  $O(n \log n)$ steps.
\end{theorem}  

\noindent \textbf{Proof:}

 Let $\vec{S^*}$ be an optimal orientation of the input star graph $S_n$.
 
 If in $\vec{S^*}$ the edges are all oriented outwards, or all oriented 
 inwards, then the algorithm will find an optimal orientation
 at the completion of the loop of statement \ref{i1}, or
of statement \ref{i2}, respectively.
 
 Otherwise at least one of the edges $(u_1,c)$ and $(c,v_1)$ takes part 
 in $\vec{S^*}$, according to Lemma \ref{l.best}. If both participate then an optimal orientation is found in step \ref{i0}.
 If not, we consider and prove the case that $(c,v_1)$ participates. 
 
 Note first of all that if there is an inward edge in $S^*$ that is oriented as
 an outward edge in $\vec{S_n}$ then flipping that edge cannot
 increase the cost of the orientation, because its outward weight does not exceed $w(c,v_1)$.
 We can assume, therefore, that all edges in $E_{out}$ participate in $S^*$. 
 Let $(u_j,c)$ be the 
 edge from $E_{in}$ with least index that participates in $S^*$, 
  $w(u_j,c)<w(u_1,c)$, i.e. all edges $\{u_i,c\}, 1\leq i<j$, are outward edges in $S^*$. 
 The cost of $S^*$ is therefore at least
 $$h_m(\vec{S^*})=w(u_j,c)+\max \{w(c,v_1),\max \{w(c,u_i):1\leq i<j)\}\}.$$
To complete the proof we observe that $h_m(\vec{S^*})$ is precisely the value of the orientation
obtained by the algorithm after flipping the edge  $\{u_{j-1},c\}$
 in iteration $k=j-1$ of the loop of step \ref{i1}.

The running time  analysis is straightforward: the reordering of step \ref{i00} takes
$O(n \log n)$, and each update of $BestCost$ in loops \ref{i1} and \ref{i2} takes 
constant time if  auxiliary variables are maintained for the values of 
$\max \{w(c,u_i):1\leq i<j)\}$ and $\max \{w(v_i,c):1\leq i<j)\}$.
 \qed
 
  \subsection{An algorithm for cost function $H_s$}
  An algorithm for cost function $H_s$ is obtained by slightly modifying Algorithm BestOrientStar-m using the following observation.
  
  \begin{lemma}
  	Only  the weights of positive weight edges contribute to the cost of an oriented star 
  	under $h_s$.
  \end{lemma} 
\begin{proof}
	The length of a path in an oriented star is no more than 2. If a path of length 2 contains a negative weight edge it is either first or last and consequently does not contribute to the cost of the path. 
\end{proof}

Here is the outline of the resulting algorithm for cost function $H_s$:
\begin{enumerate}
	\item Remove from $S_n$ every edge one of whose directions has negative weight. Denore the resulting star $S$.
	\item $\vec{S}\leftarrow \mbox{BestOrientStar-m} (S)$.
	\item Add to $\vec{S}$ each negative weigh edge removed in step 1.
\end{enumerate}