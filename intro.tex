\section{Definitions}

A \emph{bi-weighted graph} is an undirected graph $G = (V, E)$, such that each edge $\{u, v\} \in E$ has a pair of (possibly different) weights $w(u, v)$ and $w(v, u)$ associated with its two possible orientations $(u, v)$ and $(v, u)$, respectively. An \emph{orientation} of $G$ is a directed graph $\dir$ obtained from $G$ by selecting for each undirected edge $\{u, v\} \in E$ exactly one of its two possible orientations. Denote by $\mathcal{O}(G)$ the set of all orientations of $G$. 

A \emph{path} in a directed graph $\dir = (V, E)$ is a series of vertices $P = \langle v_0, v_1, \ldots, v_m\rangle$ such that for every $0 \leq i < m$, $(v_i, v_{i+1}) \in E$. Such a path is \emph{maximal} if for every $u \in V$, $(u, v_0), (v_m, u) \notin E$. The weight of a path is defined as the sum of path edge weights. Denote by $h(\dir)$ and $\tilde{h}(\dir)$ maximum weights of a path and a maximal path in $\dir$, respectively. [SZ: how should cycles be handled?]

We consider here two kinds of problems:

\begin{problem}[Minimum heaviest oriented path]
	Given a bi-weighted graph $G$, compute 
	$$
	H(G) = \min_{\dir \in \mathcal{O}(P)} h(\dir)
	$$
\end{problem}

\begin{problem}[Minimum heaviest oriented maximal path]
	Given a bi-weighted graph $G$, compute 
	$$
	\tilde{H}(G) = \min_{\dir \in \mathcal{O}(P)} \tilde{h}(\dir)
	$$
\end{problem}

