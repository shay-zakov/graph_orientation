\section{Introduction}

A \emph{edge bi-weighted graph} is an undirected graph $G = (V, E)$ in which each edge $\{u, v\} \in E$ has a pair of (possibly different) weights $w(u, v)$ and $w(v, u)$ associated with its two possible orientations $(u, v)$ and $(v, u)$, respectively. An \emph{orientation} 
of $G$ creates a directed graph $\vec{G}$ 
by selecting for each undirected edge $\{u, v\} \in E$ exactly one of its two possible orientations. Denote by $\mathcal{O}(G)$ the set of all orientations of $G$. 

The problem we address is the following.

\noindent {\bf Input}: An edge bi-weighted graph $G$.

\noindent {\bf Output}: An orientation  of $G$ that minimizes the weight of 
the heaviest of the resulting simple directed paths.

Actually, there remains an ambiguity in the specification of the cost function because it 
fails to specify what kinds of paths the minimum is to be taken over: all simple 
directed subpaths or only maximal ones. To distinguish between these two possibilities we define two measures.

\begin{definition}
	Given a directed path $\vec{P}=<v_0,\ldots,v_n>$ denote its cost
	\begin{equation}\label{eq.W}
		W(\vec{P})=\sum_{k=0}^{n-1} w(v_k,v_{k+1}).
	\end{equation}
	We define two cost measures for an oriented graph $\vec{G}$:
	\begin{align} 
		h_s(\vecG) &= \max\{W(\vec{P})\mid \vecP \mbox{ is a simple path in }\vecG \},\label{eq.hs} \\
		h_m(\vecG) &= \max\{W(\vec{P})\mid \vecP \mbox{ is a maximal simple path in }\vecG \}.\label{eq.hm}
	\end{align}
	The corresponding cost functions for orienting an undirected graph are 
	\begin{align}
		H_s(G) &= \min\{h_s(\vecG)\mid \vecG \in \mathcal{O}(G)\}, \\
		H_m(G) &= \min\{h_m(\vecG)\mid \vecG \in \mathcal{O}(G)\}.
	\end{align}
\end{definition}
Note that always $h_s(\vecG)\geq 0$, because the empty path is one of the possible simple paths appearing in equation (\ref{eq.hs}).

\noindent {\bf Example}: To illustrate the differences between the two cost functions consider
 $\vec{P}=<v_0,v_1,v_2,v_3>$ and $\vec{P_{1,2}}=<v_1,v_2>$, with $w(v_0,v_1)=2, w(v_1,v_2)=-3,w(v_2,v_3)=6$.
 Then $h_m(\vec{P})=W(\vec{P})=5, h_s(\vec{P})=6$, and 
 $h_m(\vec{P_{1,2}})=-3, h_s(\vec{P_{1,2}})=0$.
 \qed

A very useful property of $H_s$ is that it is monotone: the cost of a graph is never less than the cost of a subgraph.
\begin{lemma}\label{lem:sprop}
	Given any edge bi-weighted graph G,
	$H_{s}(G)\geq  H_{s}(G')$ for any subgraph $G'$ of $G$. 
\end{lemma}

\begin{proof}
It is clear from the definition (\ref{eq.hs}) of $h_s$ that $h_s(\vecP)\geq h_s(\vecP ')$for any subpath 
$\vecP '$ of $\vecP$.  
\end{proof}

This property sets $H_m$ apart from $H_s$,  as the above example shows:
$h_{m}(\vecP)=5 <  h_{m}(\vecP')=6$ for the subgraph $\vecP '=<v_2,v_3>$ of $\vecP$.

However, if all weights are non-negative $H_m$, too, is monotonic. In fact, in that case the two definitions coincide.

\begin{lemma}
	If all of the weights an edge-bi-weighted graph $G$ are non-negative 
	then $h_{s}( \vecG )= h_{m}( \vecG )$ for any orientation $\vecG$ of $G$.
	In particular,  $H_{s}(G)= H_{m}(G)$,  and 
	$\vecG$ is an optimal orientation of $G$ with respect to $h_{s}$
	if and only if it is optimal with respect to $ h_{m} $.
\end{lemma}
\begin{proof}
From the definitions of $H_m$ and $H_s$ it is clear that for any directed path $\vec{P}$ $h_{m}(\vec{P})\leq  h_{s}(\vec{P})$, and that the inequality is in fact an equality if all the weights 
are non-negative. This implies that $h_{s}( \vecG )= h_{m}( \vecG )$ for any orientation $\vecG$ of $G$.
\end{proof}

\bigskip
{\bf Conjecture}:
\textit{Given an edge bi-weighted graph $G$, any orientation that is optimal with 
	respect to $h_{m}$ is also optimal with respect to $h_{s} $.}

\bigskip
In the coming sections we will consider the following two problems
for various classes of graphs.

\begin{problem}[
%	Minimizing the heaviest subpath - 
	HS]
	Given a class of edge bi-weighted graphs find an algorithm to compute
	$H_s(G) $
	for any $G$
	in the class.
\end{problem}

\begin{problem}[
%	Minimizing the heaviest maximal path - 
	HM]
	Given a class of edge bi-weighted graphs find an algorithm to compute $H_m(G) $
	for any $G$
	in the class.
\end{problem}
