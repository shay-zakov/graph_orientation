
\section{An Algorithm for Linear Graphs}

%A \emph{linear graph} $P$ of length $n$ is an undirected graph with $n+1$
%vertices $V = \{v_0, v_1, \ldots, v_n\}$ and $n$ edges $E = \left\{\{v_i, v_{i+1}\} : 0\leq i < n\right\}$. 
Given a bi-weighted linear graph $P$ of length $n$ we assume that its vertices
are numbered from \textit{0} to \textit{n}, and denote the weights of
edges $(i,i+1)$ and  $(i+1,i)$ by $w(i,i+1)$ and $w(i+1,1)$, respectively.

We describe next an algorithm for finding an optimal orientation 
of a linear graph. The high level version of the algorithm makes no use of the details
of the cost function, be it $h_m$ or $h_s$. The description will therefore omit these subscripts. However, upon implementing the high level version, the difference between 
the two cost functions leads to a surprising difference in running times: it is linear 
under $h_s$ but quadratic under $h_m$.
\bigskip

{\bf Notation}:
\begin{itemize}
 	\item $P_{i, j}$ is the sub-graph of $P$ induced by the vertices $i,  \ldots, j$, and $P=P_{0,n}$. 
 	\item $\vec{P}_{i, j}$ denotes the oriented version 
 	of $P_{i, j}$ in which all edges are directed to the right (i.e. of the form $(k, {k+1})$),
 	and similarly  $\cev{P}_{i, j}$ denotes the oriented version 
 	of $P_{i, j}$ in which all edges are directed to the left.
 	\item $H_x(j)$ is the value of an optimal orientaton of $P_{0, j}$ with respect to the $h_x$ cost function, where $x = s$ or $x = m$. $H^r_x(j)$ and $H^{\ell}_x(j)$ are defined similarly, under the constraint 
 	that edge $\{j-1, j\}$ is directed either towards $j$ or towards $j-1$, respectively. Using this notation we are assuming that the original instance graph $P$ is clear from the context.
\end{itemize}

Clearly,

\begin{equation}
\label{eq:Hx}
H_x(j) = \min \left(H^r_x(j), H^\ell_x(j)\right).
\end{equation}

Consider next the set of orientations of $P_{0, j}$ in which the orientation of $\{j-1, j\}$ is directed towards $j$. In any such an orientation, there exists some unique $0 \leq i < j$ such that for every $i \leq k < j$ the edges $\{k, k+1\}$ are oriented as $(k, k+1)$, and either $i = 0$ or the edge $\{i-1, i\}$ is oriented as $(i, i-1)$. Any path in such an orientation is fully contained in either the induced orientation of the prefix $P_{0, i}$, or the induced orientation of the suffix $P_{i, j}$ (as both edges $(i, i-1)$ and $(i, i+1)$ cannot participate in a single path). In the former case, it is clear that an optimal orientation for the prefix would be one that minimizes the heaviest path weight for the edge bi-weighted subgraph $P_{0, i}$ under the constrain that $\{i-1, i\}$ is oriented as $(i, i-1)$. By definition, the weight of such an orientation is $H^\ell_x(i)$. For the suffix $P_{i, j}$ the orientation is fixed to $\vec{P}_{i, j}$, and the weight of a heaviest path in it is $h_x(\vec{P}_{i, j})$. Therefore, for a particular split index $i$ the minimum weight of an orientation as describe above would be $\min\left(H^\ell_x(i), h_x(\vec{P}_{i, j})\right)$, and considering all possible split indices we get the following:

\begin{equation}\label{eq:Hr}
H^r_x(j) = \left\{\begin{array}{ll}
0, & j = 0\\
\min_{0 \leq i < j}\left(\max\left(H^\ell_x(i), h_x(\vec{P}_{i, j})\right)\right), & \text{otherwise}\\
\end{array}\right.
\end{equation}

Symmetrically: 

\begin{equation}\label{eq:Hl}
H^\ell_x(j) = \left\{\begin{array}{ll}
0, & j = 0\\
\min_{0 \leq i < j}\left(\max\left(H^r_x(i), h_x(\cev{P}_{i, j})\right)\right), & \text{otherwise}\\
\end{array}\right.
\end{equation}

[TODO: explain how to naively compute $h_x(\vec{P}_{i, j})$ for both values of $x$, and show that all values of this form can be computed in a quadratic time].

The recursive computation described by Equations~\ref{eq:Hx} to~\ref{eq:Hl} can be efficiently implemented in a bottom-up fashion. Algorithm~\ref{algo:H} describes this implementation, utilizing two arrays $H^r[\cdot]$ and $H^\ell[\cdot]$ to store values of the forms $H^r_x(i)$ and $H^\ell_x(i)$ respectively (we use rectangle brackets to distinguish the algorithm entity from the formal one to which it corresponds), and two diagonal matrices $h^r[\cdot, \cdot]$ and $h^\ell[\cdot, \cdot]$ to store values of the forms $h_x(\vec{P}_{i, j})$ and $h_x(\cev{P}_{i, j})$, respectively.


\begin{algorithm}
	Compute all values of the form $h_x(\vec{P}_{i, j})$ and $h_x(\cev{P}_{i, j})$ using Algorithm~XXX, and store them in the corresponding matrices $h^r[\cdot, \cdot]$ and $h^\ell[\cdot, \cdot]$, resp.\;
	Allocate two $n+1$-length arrays $H^r$ and $H^\ell$, and initialize both $H^r[0]$ and $H^{\ell}[0]$ to $0$\;
	\For{$j=1$ to $n$}{
		$H^{r}[j] \gets \min_{0\leq i <j} \max \{ H^{\ell}[i], h^r[i, j]\}$\;
		$H^{\ell}[j] \gets \min_{0\leq i <j} \max \{ H^r[i], h^l[i, j]\}$\;
	}
	\Return{$\min \{H^{r}[n],\ H^{\ell}[n]\}$}\;
	\caption{BestCostPath ($P$)}
	\label{algo:H}
\end{algorithm}

The computation of all values $h_x(\vec{P}_{i, j})$ and $h_x(\cev{P}_{i, j})$ in the first line of Algorithm~\ref{algo:H} takes $O(n^2)$ as shown above. The loop in lines~3 to~5 is executed $n$ times, where the expressions computed within each loop iteration take $O(n)$ time to compute (as all values of the form $H^{\ell}[i]$, $h^r[i, j]$, $H^{r}[i]$, and $h^l[i, j]$ are pre-computed, and $O(n)$ split-points $i$ are examine in each iteration). Therefore, the overall time complexity of the algorithm is $O(n^2)$, and its space complexity is too $O(n^2)$, dictated by the $h$ matrices.


****************************************


Turning now to the running time analysis, we note first of all that the values 
$ h(\vec{P_{i, j}}), h(\cev{P_{i, j}})$ used in statements (\ref{i.1}) and (\ref{i.2})
can be computed in constant time if the cost function is $h_m$. Namely, if the $2n$ values 
$ h_m(\vec{P_{0, j}})$, $h_m(\cev{P_{0, j}})$, $1\leq j \leq n$ are precomputed in $O(n)$ time,
then
$h_m(\vec{P_{i, j}})=h_m(\vec{P_{0, j}})-h_m(\vec{P_{0, i}})
%=\sum_{k=i}^{j-1} w(k,k+1).$
$.

To prove that this is true also for the cost function $h_s$ takes more doing.

%It remains to determine the time needed for the computation of the minimum values 
%in those statements.
%On the face of it, it takes $O(n)$ time to compute one such value.
%We show now that the totality of these computations takes $O(n)$ time
%if the cost function is $h_s$.
%
%\begin{theorem}
%When the cost function used in Algorithm BestCostPath ($P$) is $h_s$ its running time is $O(n)$.
%\end{theorem}
%
%\noindent {\bf Proof}.
%According to the foregoing analysis we need to show that the total time needed for the computations of all minimum values in statements (\ref{i.1}) and (\ref{i.2}) is $O(n)$.
%The key to the proof is the following fact.
%\begin{claim}
%	Denote 
%	$$a^{\ell}[j]=\argmin_{0\leq i <j} \max \{ H_s^{\ell}[i], h_s(\vec{P_{i, j}}) \},
%   a^{r}[j]=\argmin_{0\leq i <j} \max \{ H_s^{r}[i], h_s(\vec{P_{i, j}}) \}.
%	$$
%	Then $a^{\ell}[j]\leq a^{\ell}[j+1]$, and $a^{r}[j]\leq a^{r}[j+1]$, $0\leq j <n$.
%\end{claim}
%
%\noindent The proof of the Claim rests on two observations:
%\begin{enumerate}
%	\item $h_s(\vec{P_{i, j}}$ is non-increasing as a function of $i$ and non-decreasing as a function of $j$. 
%\end{enumerate}
%\qed
%
%-----------------------------------------------------------------------------------------
%
%Let $P$ be a bi-weighted linear graph of length $n$. Denote:
%\begin{itemize}
%	\item $P_{i, j}$: the sub-graph of $P$ induced by the vertices $\left\{v_i, v_{i+1}, \ldots, v_j \right\}$. For short, denote by $P_j$ the prefix $P_{0, j}$ of $P$;
%	\item $\vec{P'} \vec{P''}$: an orientation of $P$ obtained by concatenating the orientations $\dir[P']$ and $\dir[P'']$ for some prefix $P_{k}$ and suffix $P_{k, n}$ of $P$, respectively;
%	\item $\dir[P^r]$, $\dir[P^l]$: the orientations of $P$ in which all edges are directed "rightward" (i.e. of the form $(v_i, v_{i+1})$) or "leftwards" (i.e. of the form $(v_{i+1}, v_i)$), respectively;
%	\item $\mathcal{O}^r(P) = \left\{\dir[P'] \dir[P^r]_{n-1,n} : \dir[P'] \in \mathcal{O}(P_{n-1})\right\}$ (i.e. the subset of $\mathcal{O}(P)$ containing all orientations of $P$ ending with the directed edge $(v_{n-1}, v_n)$);
%	\item $\mathcal{O}^l(P) = \mathcal{O}(P) \setminus \mathcal{O}^r(P) = \left\{\dir[P'] \dir[P^l]_{n-1,n} : \dir[P'] \in \mathcal{O}(P_{n-1})\right\}$ (i.e. the subset of $\mathcal{O}(P)$ containing all orientations of $P$ ending with the directed edge $(v_n, v_{n-1})$);
%	\item $H^r(P) = \min_{\dir[P] \in \mathcal{O}^r(P)} h(\dir[P])$;
%	\item $H^l(P) = \min_{\dir[P] \in \mathcal{O}^l(P)} h(\dir[P])$.
%\end{itemize}
%
%Before describing the computation of $H(P)$, we first describe an auxiliary computation of values of the form $h(\dir[P^r]_{i, j})$ which will be used within the computation of $H(P)$. This computation utilizes the \emph{deque} data structure to store information about $\dir[P^r]_{i, j}$ that will allow the efficient computation of $h(\dir[P^r]_{i, j})$. In addition, given the structure for the segment $\dir[P^r]_{i, j}$, the updated structure with respect to segments $\dir[P^r]_{i+1, j}$ and $\dir[P^r]_{i, j+1}$ can be computed efficiently. The computation of values of the form $h(\dir[P^l]_{i, j})$ is symmetric and we omit its explicit description. 
%
%Let $S = \langle s_0, s_1, ..., s_{m-1}\rangle$ be a series of numbers. When $m > 0$, denote by $k_S$ the index $0 \leq k < m$ so that $s_{k_S}$ is the maximum of $S$, and $k_S$ is the biggest such index if there are multiple maximal values in $S$. In other words, for every $k_S < k < m$, $s_{k_S} > s_k$. The \emph{decreasing maximum indices} $S^{\downarrow}$ of $S$ is inductively defined as follows. When $S$ is empty then $S^{\downarrow}$ is empty, and otherwise $S^{\downarrow}$ is the series of indices starting with $k_S$ and continuing with the decreasing maximum indices of the suffix $\langle s_{k_S+1}, s_{k_S+2}, ..., s_{m-1}\rangle$ of $S$. For example, for $S = \langle 5, 4, 5, 2, 3, -1, 0, 1\rangle$, $S^\downarrow = \langle 2, 4, 7\rangle$, which corresponds to the decreasing subsequence $\langle 5, 3, 1\rangle$ of $S$.
%
%Assume the decreasing maximum indices $S^\downarrow_{i, j}$ is given with respect to some interval $S_{i, j} = \langle s_i, s_{i+1}, ..., s_{j-1}\rangle$ in $S$, and consider the corresponding indices $S^\downarrow_{i, j+1}$ with respect to $S_{i, j+1} = \langle s_i, s_{i+1}, ..., s_{j}\rangle$. It is simple to assert that $S^\downarrow_{i, j+1}$ can be obtained by removing from the suffix of $S^\downarrow_{i, j}$ all indices $k$ such that $s_k \leq s_j$, and adding the index $j$ at the end of the resulting sequence. To compute $S^\downarrow_{i+1, j}$, one just remove the index $i$ from $S^\downarrow_{i, j}$ in case it is the first index in this series, and otherwise it is identical to $S^\downarrow_{i, j}$. In both cases, the time required for updating the data structure is proportional to the total number of element insertions and deletions from the index sequence.
%
%
%
%
%********************************
%
%
%In what follow, we describe a recursive computation for $H(P_j)$ for every prefix $P_j$ of $P$ (and in particular for $P = P_n$). Clearly,
%
%\begin{equation}\label{eq:H}
%H(P_j) = \min\left(H^r(P_j), H^l(P_j)\right).
%\end{equation}
%
%
%Consider an orientation $\dir[P_j] \in \mathcal{O}^r(P_j)$. By definition, there exists some $0 \leq i < j$ such that $\dir[P_j] = \dir[P'] \dir[P^r]_{i, j}$, where $\dir[P'] \in \mathcal{O}^l(P_i)$. Observe that any path in $\dir[P_j]$ cannot contain edges from both $\dir[P']$ (which is either empty when $i=0$, or otherwise ending with a leftward edge $(v_i, v_{i-1})$) and $\dir[P^r]_{i, j}$ (starting with a rightward edge $(v_i, v_{i+1})$). In particular, a heaviest path in $\dir[P_j]$ is fully contained in either $\dir[P']$ or $\dir[P^r]_{i, j}$. Therefore, $h(\dir[P_j]) = \max\left(h(\dir[P']), h(\dir[P^r]_{i, j})\right)$. An orientation that would minimize the heaviest path weight among all orientations of this form with the same $i$ value would thus have the weight $\max\left(H^l(P_i), h(\dir[P^r]_{i, j})\right)$, and we get that 
%
%\begin{equation}\label{eq:Hr}
%H^r(P_j) = \left\{\begin{array}{ll}
%0, & j = 0\\
%\min_{0 \leq i < j}\left(\max\left(H^l(P_i), h(\dir[P^r]_{i, j})\right)\right), & \text{otherwise}\\
%\end{array}\right.
%\end{equation}
%
%Symmetrically: 
%
%\begin{equation}\label{eq:Hl}
%H^l(P_j) = \left\{\begin{array}{ll}
%0, & j = 0\\
%\min_{0 \leq i < j}\left(\max\left(H^r(P_i), h(\dir[P^l]_{i, j})\right)\right), & \text{otherwise}\\
%\end{array}\right.
%\end{equation}
%
%
%Next, we describe how to efficiently compute the above recursion. 
%
%
%\begin{observation}
%	\label{obs:subpath}
%	For every $0 \leq i \leq j \leq n$, $H(P_{i, j}) \leq H(P)$, $h(P^r_{i, j}) \leq h(P^r)$, and $h(P^l_{i, j}) \leq h(P^l)$.
%\end{observation}
%
%The correctness of the observation follows from the fact that any orientation for $P$ induces an orientation for $P_{i, j}$ in which the weight of a heaviest path can only decrease. Next, consider equation~\ref{eq:Hr}, and denote by $k_P$ the maximal integer that minimizes the equation term for $n > 0$. 
%
%\begin{claim}
%	\label{clm:kp}
%	$H(P_{0, k_P}) < H(P_{0, k_P+1})$.
%\end{claim}
%
%\begin{proof}
%	By definition of $K_P$, $\max\left(H^l(P_{0, k_P}), h(P^r_{k_P, n})\right) < \max\left(H^l(P_{0, k_P + 1}), h(P^r_{k_P + 1, n})\right)$. Since $h(P^r_{k_P, n}) \geq h(P^r_{k_P+1, n})$ (from Obs.~\ref{obs:subpath}), it must be that $H(P_{0, k_P}) < H(P_{0, k_P+1})$
%\end{proof}
%
%
%\begin{claim}
%	\label{clm:kp_monotonicity}
%	For every $0 \leq j < n$, $K_{P_{0, j}} \leq K_P$.
%\end{claim}
%
%\begin{proof}
%	
%\end{proof}
