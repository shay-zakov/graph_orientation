\section{Algorithms for \textit{k}-legged spiders}\label{s.3}

\noindent {\bf Terminology}: A \textit{spider (graph)}  is a connected undirected graph in which 
there is one  vertex of degree at least 3 and all other vertices have degree at most 2.
A spider can therefore have arbitrary many legs, each of which is a path of possibly different length. 
Spiders with $k$ legs will be called $k$-legged spiders. 

\noindent {\bf Notation}: Assign the lone vertex of degree at least 3 the number 0,
and number the vertices on leg $\ell$, going outward from vertex $0$, as $(\ell,i), 1\leq i \leq n_\ell$.
For convenience  $(\ell,0)$ denotes the common vertex \textit{0}. Denote by $L^{\ell}$ the linear graph
that is leg $\ell$ of the spider, i.e. the subgraph induced by vertices $(\ell,0),\ldots, (\ell,n_{\ell})$.
Where the number of the leg is evident from the context we omit it.
For example, the weight of the directed edge from $(\ell,i)$ to $(\ell,i+1)$ will be denoted
$w_{\ell}(i,i+1)$.

Continuing with the notation of Section \ref{s.2}, 
$L^{\ell}_{i, j}$ is the sub-graph of $L^{\ell}$ induced by the vertices $(\ell,i),  \ldots, (\ell,j)$,
$\vec{L^{\ell}}_{i, j}$ is its oriented version in which all edges are directed to the right 
(from $k$ to $k+1$),
and in the oriented version $\cev{L^{\ell}}_{i, j}$ all edges are directed to the left. 

To state the algorithm for finding an optimal orientation of a given spider, we slightly modify some other notations of Section \ref{s.2}: 
$H^{\ell}_x(i,d)$ is the value of the optimal orientation of $L^{\ell}_{i,n_\ell}$
under the constraint that the edge $\{i,i+1\}$ is directed towards $i+1$ if $d=\succ$ or 
towards $i$ if $d=\prec$. We will use $\bar{d}$ to denote the direction opposite to $d$, e.g.
$\bar{\prec}=\succ$.

A possible brute force algorithm considers for each leg the two possible directions of the first edge on that leg, and all possible end- (respectively start-) points of the directed path that starts 
(respectively begins) with that edge. Denote by
 $D=(d_1,\ldots ,d_k)$ with $d_{\ell}\in\{\succ,\prec\}$
 the $k$-vector of directions of  the first edges,
  and by $J=\{j_1,\ldots j_k\}$ leg $\ell$ with $1\leq j_{\ell} \leq n_{\ell}$ 
  the $k$-vector of vertices. 
For $d_{\ell}= \prec,
d_{m}=\succ $ we denote by ${\cev{L^{\ell}}_{0, j_{\ell}}} \ast \vec{L}^{m}_{0, j_{m}}$ the directed path that starts at vertex $(\ell,j_{ell})$, passes through vertex $0$ and ends at vertex $(m,j_m)$.

\begin{enumerate}
		\item initialize $BestCost$ to $\infty$;
	\item \label{i.pp} \textbf{for} each pair of vectors $(D,J)$	\textbf{do}
\begin{enumerate}

	\item compute the maximum cost of a directed path through $0$:\\
	$$W_x(D,J)=\max \{h_x({\cev{L^{\ell}}_{0, j_{\ell}}} \ast \vec{L}^{m}_{0, j_{m}}) \mid 
	\mbox{all } \ell,m \mbox{ such that }d_{\ell}= \prec,
	d_{m}=\succ 
	\};$$
	\item optimally orienting the remainders of the legs, find the leg whose orientation is costliest:
	$H_x(D,J)=\max \{H^{\ell}_x(j_{\ell},\bar{d_{\ell}}) \mid 1\leq \ell \leq k\}$;
	\item \textbf{if} $\max\{W_x(D,J),H_x(D,J)\}\leq BestCost$ \textbf{then} update $BestCost$;
\end{enumerate}
	\item \textbf{return} an orientation whose cost is $BestCost$;
\end{enumerate}
A straightforward implementation of this algorithm is exponential in $k$. In the next section we 
present an algorithm under the $H_s$ cost function that runs in $O(n\log n +k \log k \log n)$ time.
\subsection{An algorithm for cost function $H_s$}
Here we take a somewhat different approach to guide the design of the algorithm, namely we view
an (optimal) orientation as consisting of an orientation of the central part,
the star-like body of the spider, and an orientation of each of the remainders of the legs.
%of the spider that begin at the first joint (i.e. a vertex 
%that has outdegree 2 and indegree 0, or the reverse). 
Given  a weight bound, $\gamma$, Algorithm \ref{a.spiderbody} 
describes how to obtain the bi-weighted body of the spider by amputating each leg
at the joint (a vertex 
that has outdegree 2 and indegree 0, or the reverse) that is closest to the body under the constraint that the part that is
amputated does not exceed $\gamma$. Note that the joint at
which the amputation is performed can differ between the outward and inward direction 
from the body. The data the algorithm uses to determine that joint are
$H^{\ell}_s(j,\prec)$ and $H^{\ell}_s(j,\succ)$, as defined above. 
These data are computed in the main Algorithm
\ref{a.spiders} using BestCostLinear$_s$($L^{\ell}_{1,n_\ell}$).

%\renewcommand{\forcond}{$\ell$ \KwTo\Range{$k$}}
\begin{algorithm}
	\newcommand{\forcond}{$\ell=1$ \KwTo $k$.}
	\KwIn{a bi-weighted $k$-legged spider $Sp$, and a weight $\gamma\geq 0$}
	\KwOut{the bi-weighted star $S$ 
obtained from $Sp$ 
		by amputating each leg $L^{\ell}$ (separately in each direction) as close as possible to the body provided the amputated part has cost at most $\gamma$.}	
	\KwData{$H^{\ell}_s(j,\prec)$ and $H^{\ell}_s(j,\succ)$, $1\leq j\leq n_{\ell}$, $1\leq \ell \leq k$.}
	\For{\forcond}
			{
				let $j_{\prec}$ be the smallest $j\geq 1$ such that  $H^{\ell}_s(j,\prec)\leq \gamma$,
				and let $j_{\succ}$ be the smallest $j\geq 1$ such that  $H^{\ell}_s(j,\succ)\leq \gamma$\;
				set $w(0,\ell)$ to $h_s(\vec{L}^{\ell}_{0,j_{\prec}})$, and set $w(\ell,0)$ to 
				$h_s(\cev{L}^{\ell}_{0,j_{\succ}})$
			}
			let $S$ be the star whose weights are $w(0,\ell)$ and $w(\ell,0)$, $1\leq \ell \leq k$\;
	{\Return $S$}
	\caption{Algorithm SpiderBody$_s (Sp,\gamma)$}
\label{a.spiderbody}
\end{algorithm}
\begin{algorithm}
	\newcommand{\forcond}{$\ell=1$ \KwTo $k$}
	\KwIn{a bi-weighted $k$-legged spider $Sp$}
	\KwOut{the cost of an optimal orientation of $Sp$ under $H_s$}	
	\For {$\ell= 1$ to $k$}
	{  
%		set $H^{\ell}_s(n_{\ell},\prec)$ and $H^{\ell}_s(n_{\ell},\succ)$ to $0$\;
		 \For {$j= 1$ to $n_{\ell}$}
		{
		compute $H^{\ell}_s(j,\prec)$ and $H^{\ell}_s(j,\succ)$ using BestCostLinear$_s$($L^{\ell}_{1,n_\ell}$);
		}
	}
	let $\gamma_i$, $1\leq i \leq 2n$ be the $2n$ values $H^{\ell}_s(j,\prec)$ and $H^{\ell}_s(j,\succ)$, 
	$1\leq \ell \leq k$, $1\leq j \leq n_{\ell}$, sorted in non-increasing order\;
	set $S^1$ to SpiderBody($Sp,\gamma_1$) ~~~~   \tcc*[h]{smallest possible body}\;
	\lIf{\emph{BestCostStar}$_s(S^1) \geq \gamma_1$}{set $BestCost$ to BestCostStar$_s$($S^1$) }
    \Else{
    find $i$ such that $\mbox{BestCostStar}_s(S^i) <\gamma_i $,  $S^i$= SpiderBody($Sp,\gamma_i$),
    and BestCostStar$_s(S^{i+1})\geq \gamma_{i+1}$, $S^{i+1}$= SpiderBody($Sp,\gamma_{i+1}$)
    \;
    \lIf{$\gamma_{i}\leq \mbox{BestCostStar}_s(S^{i+1})$}
{set $BestCost$ to $\gamma_{i}$}
\lElse{set $BestCost$ to BestCostStar$_s(S^{i+1})$}    
}
\Return $BestCost$;
	\caption{Algorithm BestCostSpider$_s (Sp)$}
	\label{a.spiders}
\end{algorithm}

\begin{theorem}
Algorithm BestCostSpider$_s (Sp)$ returns an optimal orientation of $Sp$ in $O((n + k \log k )\log n)$ time.
\end{theorem}
\begin{proof}
	Let's first explain the rationale behind Algorithm BestCostSpider.
	Under $h_s$ (but not under $h_m$) $H^{\ell}_s(j,\prec)$ and $H^{\ell}_s(j,\succ)$
	are non-increasing functions of $j$. Consequently,
	the $j_{\prec}$ found for $\ell$ in SpiderBody$_s (Sp,\gamma)$ for $\gamma=\gamma_i$
	is no larger than the one found for $\gamma=\gamma_{i+1}$, and 
	accordingly $w(0,\ell)$ is no larger for $\gamma=\gamma_i$
	than it is for $\gamma=\gamma_{i+1}$. Denoting 
	$\beta_i=\mbox{BestCostStar}(S^i)$ with $S^i$= SpiderBody($Sp,\gamma_i$)
	as in Algorithm \ref{a.spiders}, it follows that $\beta_i\leq \beta_{i+1}$.
	
	In essence, Algorithm BestCostSpider examines the sequence of orientations
	corresponding to $\gamma_i$, $0\leq i\leq 2n$, in order to
	find the one of minimum value. The cost of orientation $i$ is
    $\max\{\gamma_i,\beta_i\}$. Note that $\gamma_i$ is non-increasing,
    with $\gamma_{2n}=0$, while $\beta_i$ is non-negative and
    non-decreasing (under $h_s$). The desired minimum is therefore achieved either for $i=1$,
	when the least-cost body is at least as costly as the most-cost leg, or for some $i$
	such that $1< i \leq 2n$. 
%	Note that $\gamma_{2n}=0$, because 
%	$H^{\ell}_s(n_\ell,\succ)=H^{\ell}_s(n_\ell,\prec)=0$, while $\sigma_{2n}\geq 0$.
Thus, if the minimum is not achieved for $i=1$, there necessarily is an $i$ such that 
$\gamma_i>\beta_i$ while $\gamma_{i+1}\leq \beta_{i+1}$, and the minimum value is 
$\min \{\gamma_i,\beta_{i+1}\}$.
	
Turning to the proof, assume that  $\tildeb{Sp}^*$ is an optimal orientation of $Sp$. 
We want to show that 
	the algorithm considers at some point a valid orientation $\tildeb{Sp}$ that satisfies 
	$h_s(\tildeb{Sp})\leq h_s(\tildeb{Sp}^*)$. The proof
	uses the following notation.
	\begin{itemize}

		\item $d_{\ell}$ is the direction of the edge $\{0,1\}$ on leg $\ell$ of $\tildeb{Sp}^*$.
		\item $j_{\ell}$ is the largest $j$ such that all edges $\{j,j+1\}$, $0\leq j\leq j_{\ell}-1$,
		on leg $\ell$ of $\tildeb{Sp}^*$ have the direction $d_{\ell}$.
		\item $\delta=\max_{1\leq \ell \leq k} H^{\ell}_s(j_{\ell},\bar{d_{\ell}})$ is the maximum cost
		  of all the amputated legs.
		\item $\tildeb{S}^*$ is the star-like body of $\tildeb{Sp}^*$, i.e. the star 
		 whose edge $\{0,\ell\}$ has direction $d_{\ell}$, and 
		  weight $h_s(\vec{L}^{\ell}_{0,j_{\ell}})$ if $d_\ell =\succ$ or weight
		  $h_s(\cev{L^{\ell}}_{0,j_{\ell}})$ if $d_\ell =\prec$.
	\end{itemize}
Note, first of all, that $h_s(\tildeb{Sp}^*)=\max \{\delta, h_s(\tildeb{S}^*)  \}$.
Let $i$ be such that $\delta=\gamma_i$, denote $S^i$= SpiderBody($Sp,\gamma_i$),
and let $j'_\ell(d)$ be the smallest $j\geq 1$ such that  $H^{\ell}_s(j,\bar{d})\leq \gamma_i$,
$1\leq \ell \leq k$, $d \in\{\succ, \prec\}$. Then $j'_{\ell}(d_{\ell})\leq j_\ell$.
Looking now at $\tildeb{S}^*$, if the direction of edge $\{0,\ell\}$ is $\succ$
then the weight of this edge, $h_s(\vec{L}^{\ell}_{0,j_{\ell}})$, is at least the weight of
the same edge in $S^i$, $h_s(\vec{L}^{\ell}_{0,j_{\ell}(\succ)})$,
because $\vec{L}^{\ell}_{0,j'_{\ell}(\succ)}$ is a subpath of $\vec{L}^{\ell}_{0,j'_{\ell}}$. The same holds, of course, if the direction of edge $\{0,\ell\}$ is $\prec$. Consequently, the cost of that orientation $\tildeb{S^i}$  of $S^i$ 
in which all edges have the same orientation as in $\tildeb{S}^*$ is no more than
$h_s(\tildeb{S}^*)$, so that $\beta_i=$BestCostStar$_s(S^{i})\leq h_s(\tildeb{S}^*)$.

We conclude that amputating $Sp$ at the joints $j_\ell(d), 1\leq k, d\in \{\succ,\prec \}$,
and optimally orienting the star-like body and the amputated legs results in an orientation
$\tildeb{Sp}$ that satisfies 
$h_s(\tildeb{Sp})=\max \{\gamma_i, \beta_i\}\leq \max \{\delta, h_s(\tildeb{S}^*) =h_s(\tildeb{Sp}^*)$.

The algorithm uses $O(n)$ time for the invocations of BestCostLinear$_s$
in statement 3, $O(n \log n)$ time to sort the $2n$ values $\gamma_i$. Using binary 
search on these sorted values to find an $i$ satisfying $\beta_i<\gamma_i$ and 
$\beta_{i+1} \geq \gamma_{i+1}$ involves $O(\log n)$ invocations of  
BestCostStar$_s$ for a total of $O(k\log k \log n)$ time.
\end{proof}
