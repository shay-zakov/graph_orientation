\section{Algorithms for \textit{k}-legged spiders}\label{s.3}

\noindent {\bf Terminology}: A \textit{spider (graph)}  is a connected undirected graph in which 
there is one  vertex of degree at least 3 and all other vertices have degree at most 2.
A spider can therefore have arbitrary many legs, each of which is a path of possibly different length. 
Spiders with $k$ legs will be called $k$-legged spiders. 

\noindent {\bf Notation}: Assign the lone vertex of degree at least 3 the number 0,
and number the vertices on leg $\ell$, going outward from vertex $0$, as $(\ell,i), 1\leq i \leq n_\ell$.
For convenience  $(\ell,0)$ denotes the common vertex \textit{0}. Denote by $L^{\ell}$ the linear graph
that is leg $\ell$ of the spider, i.e. the subgraph induced by vertices $(\ell,0),\ldots, (\ell,n_{\ell})$.
Where the number of the leg is evident from the context we omit it.
For example, the weight of the directed edge from $(\ell,i)$ to $(\ell,i+1)$ will be denoted
$w_{\ell}(i,i+1)$.

Continuing with the notation of Section \ref{s.2}, 
$L^{\ell}_{i, j}$ is the sub-graph of $L^{\ell}$ induced by the vertices $(\ell,i),  \ldots, (\ell,j)$,
$\vec{L^{\ell}}_{i, j}$ is its oriented version in which all edges are directed to the right 
(from $k$ to $k+1$),
and in the oriented version $\cev{L^{\ell}}_{i, j}$ all edges are directed to the left. 

To state the algorithm for finding an optimal orientation of a given spider, we slightly modify some other notations of Section \ref{s.2}: 
$H^{d}_x(\ell,i)$ is the value of the optimal orientation of $L^{\ell}_{i,n_\ell}$
under the constraint that the edge $\{i,i+1\}$ is directed towards $i+1$ if $d=\succ$ or 
towards $i$ if $d=\prec$. We will use $\bar{d}$ to denote the direction opposite to $d$, e.g.
$\bar{\prec}=\succ$.

\subsection{A unified algorithm for spiders}
In designing the algorithm we view
an orientation of the spider as generated by making the following choices.
\begin{enumerate}
	\item On each leg $\ell$ and for each directon $d$ pick an amputation joint (vertex) $j(\ell,d)$ at which to 
	amputate the leg in direction $d$,
	leaving both a directed stump attached to the central vertex, and a severed limb.
	\item  Construct a star with $k$ leaves 
	with edge weight to the	leaf $\ell$ equal to $h_x(\vec{L^{\ell}}_{0, j_\ell})$,
	and edge weight from the leaf equal to $h_x(\cev{L^{\ell}}_{ j_\ell,0})$,
	in essence contracting stump $\ell$ of the spider body to an edge connected to leaf $\ell$. Optimally orient the resulting star, denote the cost of this orientation by $\beta$, and denote
	its direction of orientation of stump $\ell$ by $d_\ell$. 
	\item Orient each severed limb $\ell$ optimally under the constraint that at 
	the amputation joint $j_\ell$ the direction of the severed limb is opposite to that 
	of the directed stump. In our notation the cost of this orientation
	is $H^{d}_x(\ell,j_\ell)$, with $d=\bar{d_\ell}$.
	Denote the 
	maximum orientation cost of any severed limb by $\lambda= \max_{\ell}H^{\bar{d_\ell}}_x(\ell,j_\ell)$.
	Then  the total cost of the spider orientation is $\max\{\beta,\lambda\}$.
\end{enumerate}
We will show that rather than examining all possible $k$-tuples of amputation joint indices, $(j_1,\ldots,j_k)$, it suffices to examine only those tuples resulting from the following procedure, at most $2n$
in number.
\begin{enumerate}
	\item Denote 
	\begin{alignat}{3}
	\sigma(\ell,\succ,j)&=h_x(\vec{L^{\ell}}_{0, j}),\sigma(\ell,\prec,j)&=&h_x(\cev{L^{\ell}}_{0, j}),\\
	\gamma(\ell,\succ,j)&=H^{\succ}_x(\ell,j),\ \gamma(\ell,\prec,j)&=&H^{\prec}_x(\ell,j).
	\end{alignat}
	For each $\ell$, $1\leq \ell\leq k$, and $d \in \{\succ,\prec\}$ 
	use the methods described in Section \ref{s.2} to compute all 
	 pairs $(\sigma(\ell,d,j), \gamma(\ell,d,j))$, $1\leq j \leq n_\ell$.
	\item We will say that an amputation joint $j$ is $(\ell,d)$-irrelevant if there is a $j'$ such that both 
	$\sigma(\ell,d,j')\leq \sigma(\ell,d,j)$ and $\gamma(\ell,d,j') \leq \gamma(\ell,d,j)$.
	Delete each $(\ell,d)$-irrelevant $j$ by deleting both $\sigma(\ell,d,j)$, 
	and $\gamma(\ell,d,j)$. Denote by $N_{\ell,d}$ the number of remaining 
	$(\ell,d)$-relevant joints. Note that an irrelevant amputation joint in a given orientation 
	can be replaced by a relevant amputation joint without increasing the cost of the orientation.
	\item For each $\ell$ and $d$ sort the $(\ell,d)$-relevant joints in non-increasing order
	of $\gamma(\ell,d,j)$. Denote the reordering $j_r$, $1\leq r \leq N_{\ell,d}$,
	i.e. $\gamma(\ell,d,j_r)\geq \gamma(\ell,d,j_{r+1})$. Note that 
	$\sigma(\ell,d,j_r)\leq \sigma(\ell,d,j_{r+1})$.
	\item Merge all sorted sequences $\gamma(\ell,d,j_r)$ into one grand sorted 
	non-increasing sequence
	and denote it $\gamma_i$, $1\leq i \leq N$, $N=\sum_{\ell, d} N_{\ell, d}$.
	\item For each $\gamma_i$ in turn do: 
 for each $\ell$ and $d$, 
 use the values $H^{d}_x(\ell,j_r)$ to determine the joint 
		$j_\rho$, $\rho=\rho(\gamma_i,\ell,d)$ at which to amputate leg $\ell$ so that the cost of the severed limb
	is as close as possible to $\gamma_i$ without exceeding it. 
\end{enumerate}

%\renewcommand{\forcond}{$\ell$ \KwTo\Range{$k$}}
\begin{algorithm}[!ht]
	\newcommand{\forcond}{$\ell=1$ \KwTo $k$}
	\KwIn{a bi-weighted $k$-legged spider $Sp$, and a weight $\gamma\geq 0$.}
	\KwOut{the bi-weighted star $S$ 
obtained from $Sp$ 
		by amputating each leg $L^{\ell}$ (separately in each direction) so that the severed limb has maximal cost without exceeding $\gamma$.}	
	\KwData{for all $d \in \{\succ,\prec\}$ and $1\leq \ell \leq k$, the 
		$N_{\ell,d}$ $(\ell,d)$-relevant amputation
		joints $j_r$, sorted in non-increasing order of $H^{d}_x(\ell,j_r)$.
	}
	\For{\forcond}
			{\For{$d \in \{\prec,\succ\}$}{
				let $\rho=\rho(\gamma,\ell,d)$ be the smallest $r\geq 1$ such that  $H^{d}_x(\ell,j_r)\leq \gamma$;
%				and let $j_{\succ}$ be the smallest $j\geq 1$ such that  $H^{\ell}_s(j,\succ)\leq \gamma$
			}
				set $w(0,\ell)$ to $h_x(\vec{L}^{\ell}_{0,j_\rho})$, and set $w(\ell,0)$ to 
				$h_x(\cev{L}^{\ell}(0,j_\rho)$;
			}
			let $S$ be the star whose weights are $w(0,\ell)$ and $w(\ell,0)$, $1\leq \ell \leq k$\;
	{\Return $S$}
	\caption{Algorithm SpiderBody$_x (Sp,\gamma)$}
\label{a.spiderbody}
\end{algorithm}

\begin{algorithm}[!ht]
	\newcommand{\forcond}{$\ell=1$ \KwTo $k$}
	\KwIn{a bi-weighted $k$-legged spider $Sp$}
	\KwOut{an optimal orientation of $Sp$ under $H_s$}	
	\For {$\ell= 1$  \KwTo $k$}
		{\For {$d \in \{\succ,\prec\}$}  
		  {\lFor {$j= 1$  \KwTo $n_{\ell}$}
		            {
		            compute $H^{d}_x(\ell,j)$ %and $H^{\succ}_x(\ell,j)$
	                 }
		  remove all those $H^{d}_x(\ell,j)$ for which $j$ is $(\ell,d)$-irrelevant \;
		  sort the remaining relevant $j_r$ in non-increasing order of  $H^{d}_x(\ell,j)$ , i.e. so that 
		  $H^{d}_x(\ell,j_r) \geq H^{d}_x(\ell,j_{r+1})$, $1\leq r \leq N_{\ell,d}$;		
		}
      }
	merge  $H^{d}_x(\ell,j_r)$, $1\leq \ell \leq k$, $d \in \{\succ,\prec\}$,
	$1\leq r \leq N_{\ell,d}$,  into one non-increasing sequence
	$\gamma_i$, $1\leq i \leq N$, $N=\sum_{\ell,d}N_{\ell,d}$\;
	set $BestCostSofar$ to $\infty$\;
	\For {$i=1$ \KwTo $N$}
	{  
%	set $S^i$ to SpiderBody($Sp,\gamma_i$),
%	%~~~~   \tcc*[h]{smallest possible body}
%	and set $\tildeb{S^i}$ to BestOrientStar$_x(S^i)$\;
$S^i \leftarrow$ SpiderBody($Sp,\gamma_i$), 
$\tildeb{S^i}\leftarrow$ BestOrientStar$_x(S^i)$,
$\beta_i \leftarrow h_x(\tildeb{S^i})$\;
	\tcp{construct the orientation $\tildeb{Sp^i}$ induced by $\tildeb{S^i}$ }
	initialize $\tildeb{Sp^i}$ to $\tildeb{S^i}$\;
	\For{\forcond}
	{let $d$ be direction of edge $\{0,\ell\}$ in $\tildeb{Sp^i}$, let 
	$j$ be the joint at which SpiderBody($Sp,\gamma_i$) amputated leg $\ell$ in direction $d$,
	let $\tildeb{L^d}_{0,j}$ be $\vec{L}_{0,j}$ or $\cev{L}_{0,j}$ if $d=\succ$ or $\prec$,
	and let $\tildeb{L}_{j,n_\ell}$ be an optimal orientation of $L_{j,n_\ell}$ 
	with cost $\lambda_i(\ell)=H^{\bar{d}}_x (\ell,j)$\;
	replace edge $\{0,\ell\}$ in $\tildeb{S^i}$ by the composition of 
	$\tildeb{L^d}_{0,j}$ and $\tildeb{L}_{j,n_\ell}$
	\;
	}

$\lambda_i \leftarrow \max_\ell \lambda_i(\ell)$, $h_x(\tildeb{Sp^i})=\max \{\beta_i,\lambda_i\}$\;

	\If{$h_x(\tildeb{Sp^i}) <BestCostSofar$}{ $BestCostSofar \leftarrow h_x(\tildeb{Sp^i})$ and
		 $\tildeb{Sp} \leftarrow \tildeb{Sp^i}$;}
}
\Return $\tildeb{Sp}$.
	\caption{Algorithm BestOrientSpider$_s (Sp)$}
	\label{a.spiders}
\end{algorithm}

\newpage
\begin{theorem}
Algorithm BestOrientSpider$_x (Sp)$ returns an optimal orientation of $Sp$.
BestOrientSpider$_s$ runs in $O(n \log k)$ time,
whereas BestOrientSpider$_m$ runs in $O(n\log n)$ time.
\end{theorem}
\begin{proof}
	It is fairly straightforward to verify that the orientation $\tildeb{Sp^i}$
	constructed in iteration $i$ by
	Algorithm BestCostSpider$_x$ is a valid orientation of $Sp$.
%	examines the sequence of orientations
%	corresponding to $\gamma_i$, $0\leq i\leq N$, in order to
%	find one of minimum cost. 
% these $\gamma_i$ constitute all potential 
%	values of $\lambda$.
		Let $\tildeb{Sp}^*$ be an optimal orientation of $Sp$. 
We will prove that 
there is an $i$ such that $\tildeb{Sp^i}$ satisfies 
	$h_x(\tildeb{Sp^i})\leq h_x(\tildeb{Sp}^*)$. The proof
	uses the following notation in connection with $\tildeb{Sp}$.
	\begin{itemize}
		\item $d^*_{\ell}$ is the direction of the edge $\{0,1\}$ on leg $\ell$ of $\tildeb{Sp}^*$.
		\item the amputation joint of leg $\ell$, $j^*(\ell)$, is the largest $j$ such 
		that all edges $\{i,i+1\}$, $0\leq i\leq j^*(\ell)-1$,
		on leg $\ell$ have the direction $d^*_{\ell}$.
		\item $\tildeb{S}^*$ is the star-like body of $\tildeb{Sp}^*$, i.e. the star 
		 whose edge $\{0,\ell\}$ has direction $d^*_{\ell}$, and 
		  weight $h_x(\vec{L}^{\ell}_{0,j^*(\ell)})$ or $h_x(\cev{L^{\ell}}_{0,j^*(\ell)})$ if $d^*_\ell =\succ $ or $\prec$.
		 \item $\lambda^*=\max_{1\leq \ell \leq k} H^{\bar{d^*_\ell}}_x(\ell,j^*(\ell))$ is 
		 the maximum cost among the costs of the limbs that are to be severed.
	\end{itemize}
Note that $h_x(\tildeb{Sp}^*)=\max \{\lambda^*, h_x(\tildeb{S}^*)  \}$.
Without loss of generality it can be assumed that each $j^*(\ell)$ is a 
$(\ell,d)$-relevant amputation joint
(if it is irrelevant it can be replaced by a relevant one without increasing the cost of the orientation), so that $j^*(\ell)=j_{r^*}$,
 for some $1\leq r^*\leq N_{\ell, d^*_{\ell}}$.

Let's look at $i$ such that $\gamma_i=\lambda^*$. 
Let $S^i$= SpiderBody($Sp,\gamma_i$) and let $j_\rho$ with
$\rho=\rho(\gamma_i,\ell,d^*_{\ell})$ be the amputation joint chosen by SpiderBody($Sp,\gamma_i$) for leg $\ell$ and direction $d^*_{\ell}$.
This choice of $\rho$ ensures that if $r$ is such that  
$H^{d^*_{\ell}}_x(\ell,j_r)\leq \gamma_i$
then $j_\rho\leq j_r$. Hence $j_\rho\leq j_{r^*}$, so that the weight assigned to the edge 
$\{0,\ell\}$ in the direction $d^*_{\ell}$ by SpiderBody($Sp,\gamma_i$) does not
exceed $h_x(\vec{L}_{0,j_{r^*}})$ if $d^*_{\ell}=\succ$ or 
$h_x(\cev{L}_{0,j_{r^*}})$ if $d^*_{\ell}=\prec$, i.e. the weight of each directed edge 
in $\tildeb{S}^*$ is at least as large as the weight of the edge with the same direction
in $\tildeb{S}^i$. Consequently $h_x(\tildeb{S}^*)\geq h_x(\tildeb{S}^i)$.

This proves that $h_x(\tildeb{Sp}^i)\leq \max\{h_x(\tildeb{S}^i),\gamma_i \}\leq h_x(\tildeb{Sp}^*)$.

To analyze the running times for the two cost functions, let us look at the times expended
in each part of the algorithm.
\begin{enumerate}
	\item The part consisting of lines 1-5 does the following for each $d$ and $\ell$:\\
	it computes all values $H^{d}_x(\ell,j)$, deletes the irrelevant ones, and sorts the remaining ones. 
	 
	 In case of $h_s$ this only takes $O(n)$ time because the linear graph algorithm 
	 runs in linear time, and the values $H^{d}_x(\ell,j)$ it computes are already sorted, 
	 for each  $d$ and $\ell$. In case of $h_m$, the linear graph algorithm takes $O(n\log n)$
	 time and the other computations do not take more time.
	\item Line 6 is stated separately for clarity, but is computed in tandem with the loop
	of lines 8-16, one $\gamma_i$ at a time, by merging the $2k$ sorted 
	sequences $H^{d}_x(\ell,j)$, $1\leq \ell \leq k$, $d \in \{\succ,\prec\}$. 
    To this end a priority queue holds the  $2k$ current lowest values and at each increase  
    of $i$ the minimum value is extracted, its  $\ell$, $d$ and $j$ are noted, and 
    the value $H^{d}_x(\ell,j+1)$ is inserted. Overall this takes $O(n \log k)$ time.
    
    The computation of the optimal orientation of the spider body is stated in line 9
    using BestOrientStar$_x(S^i)$ for
    clarity, although for efficiency it should be implemented rather as an update of the star
    of the optimal orientation when a single weight is increased. It is shown in Subsection
    \ref{s.star.update} that such an update takes $O(\log k)$ time. This
    establishes the stated running times.
\end{enumerate}

\end{proof}
